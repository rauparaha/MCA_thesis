\chapter[Durable equilibria]{Equilibria of the durable goods model}
\label{cha:equil-durable-goods}

\section{Time-consistency without precommitment}
\label{sec:time-cons-equil}
To solve this game by maximising the Bellman function
\eqref{tc:bell} with respect to $p^t$ first differentiate it with
respect to that variable to give the first-order condition
\begin{gather}  \label{tc:foc}
    \del{V^t}{p^t} = w^t_2 + w^t_3 f^{t+1}_1 + \delta V^{t+1}_1 = 0 \\
    \intertext{which rearranges to give} \label{tc:Bellval}
    V^{t+1}_1 = - \frac{w^t_2 + w^t_3 f^{t+1}_1}{\delta}.
\end{gather}
Now, to find the envelope condition that holds for the optimal
strategy, differentiate $V^t$ with respect to the state variable
$p^{t-1}$:
\begin{equation}
    V^t_1 = w^t_1 + f^t_1 \bigl( w^t_2  + w^t_3 f^{t+1}_1 + \delta
    V^{t+1}_1 \bigr)
\end{equation}
Since, at the optimal point, the first order condition from
\eqref{tc:foc} holds, we know that $w^t_2 + w^t_3 f^{t+1}_1 + \delta
V^{t+1}_1 = 0$ and the envelope condition is thus
\begin{equation}    \label{tc:envelope}
    V^t_1 = w^t_1
\end{equation}
Now substituting the derivative of the Bellman value function from
\eqref{tc:envelope} into equation \eqref{tc:Bellval} gives equation
\eqref{tc:mpe}.


\section{MPE of the taxation game}
\label{sec:mpe-taxation-game}
\subsubsection{The regulator's necessary condition} The first order
condition for the regulator is derived by differentiating the
regulator's Bellman equation with respect to $\tau^t$ and setting it
equal to zero.
\begin{equation}    \label{reg:foc}
    \del{W^t}{\tau^t} = w^t_3 g^{t+1}_2 + \delta W^{t+1}_2 = 0
\end{equation}
Thus the Bellman value function for the regulator is characterised
by
\begin{equation}    \label{reg:Bellval}
    W^{t+1}_2 = -\frac{w^t_3 g^{t+1}_2}{\delta}.
\end{equation}
To solve for the generalised Euler-Lagrange equations which
characterise the equilibrium strategies of the regulator and the
monopolist it is necessary to find the envelope conditions in
addition to the first-order condition. These are found by
differentiating the Bellman value function with respect to the
strategies:
\begin{align}
    W^t_1 &= w^t_1 + g^t_1 \bigl( w^t_2 + w^t_3 g^{t+1}_1 + \delta
    W^{t+1}_1 \bigr) + f^t_1 \bigl( w^t_3 g^{t+1}_2 + \delta
    W^{t+1}_2 \bigr) \\
    W^t_2 &= g^t_2 \bigl( w^t_2 + w^t_3 g^{t+1}_1 + \delta W^{t+1}_1
    \bigr) + f^t_2 \bigl( w^t_3 g^{t+1}_2 + \delta
    W^{t+1}_2 \bigr)
\end{align}
Since we know from the first order condition, \eqref{reg:foc}, that
$w^t_3 g^{t+1}_2 + \delta W^{t+1}_2 = 0$ in optimality it follows
that the envelope conditions for the regulator's optimal strategy
are
\begin{align}
    W^t_1 &= w^t_1 + g^t_1 \bigl( w^t_2 + w^t_3 g^{t+1}_1 + \delta
    W^{t+1}_1 \bigr)    \label{reg:envelope1} \\
    W^t_2 &= g^t_2 \bigl( w^t_2 + w^t_3 g^{t+1}_1 + \delta W^{t+1}_1
    \bigr).     \label{reg:envelope2}
\end{align}
From these conditions and the first order conditions it is possible
to solve for the Euler-Lagrange equations  by eliminating the
unknown Bellman value functions. First substituting
\eqref{reg:Bellval} into \eqref{reg:envelope2} gives
\begin{align}
    -\frac{w^{t-1}_3 g^t_2}{\delta} &= g^t_2 \bigl( w^t_2 + w^t_3 g^{t+1}_1 + \delta W^{t+1}_1
    \bigr) \\
    \intertext{which simplifies to}
    W^{t+1}_1 &= -\frac{w^{t-1}_3 + \delta \bigl( w^t_2 + w^t_3 g^{t+1}_1 \bigr) }{\delta^2}
    \label{reg:dWt+1dpt}
\end{align}
Now substituting \eqref{reg:dWt+1dpt} into \eqref{reg:envelope1} to
eliminate the unknown Bellman value function derivative gives
\begin{multline}
    -\frac{w^{t-2}_3 + \delta \bigl( w^{t-1}_2 + w^{t-1}_3 g^t_1 \bigr) }{\delta^2} = w^t_1 \\
    + g^t_1 \Biggl( w^t_2 + w^t_3 g^{t+1}_1 + \delta
    \biggl(-\frac{w^{t-1}_3 + \delta \bigl( w^t_2 + w^t_3 g^{t+1}_1 \bigr) }{\delta^2} \biggr) \Biggr)
\end{multline}
which, shifted forward two period, simplifies to give equation
\eqref{reg:euler}.

\subsubsection{The monopolist's necessary condition} The first order
condition for the monopolist is derived by differentiating the
monopolist's Bellman equation with respect to $p^t$ and setting it
equal to zero.
\begin{equation}    \label{mon:foc}
    \del{\Pi^t}{p^t} = \pi^t_2 + \pi^t_3 g^t_1 + \delta \Pi^{t+1}_1
    = 0
\end{equation}
The Bellman value function for the monopolist can then be
characterised by the equation
\begin{equation}    \label{mon:Bellval}
    \Pi^{t+1}_1 = - \frac{\pi^t_2 + \pi^t_3 g^t_1}{\delta}
\end{equation}
Now finding the envelope conditions for the monopolist as was done
for the regulator:
\begin{align}
    \Pi^t_1 &= \pi^t_1 + g^t_1 \bigl( \pi^t_2 + \pi^t_3 g^{t+1}_1 + \delta \Pi^{t+1}_1 \bigr)
    + f^t_1 \bigl( \pi^t_3 g^{t+1}_2  + \pi^t_4 + \delta \Pi^{t+1}_2
    \bigr) \\
    \Pi^t_2 &= g^t_2 \bigl( \pi^t_2  + \pi^t_3 g^{t+1}_1 + \delta \Pi^{t+1}_1
    \bigr)
    + f^t_2 \bigl( \pi^t_3 g^{t+1}_2 + \pi^t_4 + \delta \Pi^{t+1}_2
    \bigr)
\end{align}
From the first order condition, \eqref{mon:foc}, $\pi^t_2 + \pi^t_3
g^t_1 + \delta \Pi^{t+1}_1 = 0$ and, hence, the envelope conditions
for this problem are:
\begin{align}
    \Pi^t_1 &= \pi^t_1 + f^t_1 \bigl( \pi^t_3 g^{t+1}_2  + \pi^t_4 + \delta \Pi^{t+1}_2
    \bigr)  \label{mon:envelope1}\\
    \Pi^t_2 &= f^t_2 \bigl( \pi^t_3 g^{t+1}_2 + \pi^t_4 + \delta \Pi^{t+1}_2
    \bigr)  \label{mon:envelope2}
\end{align}
Now to solve, first substitute \eqref{mon:Bellval} into
\eqref{mon:envelope1} to give
\begin{equation}
    - \frac{\pi^{t-1}_2 + \pi^{t-1}_3 g^{t-1}_1}{\delta} = \pi^t_1 + f^t_1 \bigl( \pi^t_3 g^{t+1}_2  + \pi^t_4 + \delta
    \Pi^{t+1}_2 \bigr)
\end{equation}
which rearranges to
\begin{equation}    \label{mon:dPidtaut}
    \Pi^{t+1}_2 = - \frac{\pi^{t-1}_2 + \pi^{t-1}_3 g^{t-1}_1 + \delta \pi^t_1}{\delta^2 f^t_1}
    - \frac{\pi^t_3 g^{t+1}_2 + \pi^t_4}{\delta}.
\end{equation}
Now substituting \eqref{mon:dPidtaut} into \eqref{mon:envelope2}
eliminates the unknown Bellman value function derivative:
\begin{multline}
    - \frac{\pi^{t-2}_2 + \pi^{t-2}_3 g^{t-2}_1 + \delta \pi^{t-1}_1}{\delta^2 f^{t-1}_1}
    - \frac{\pi^{t-1}_3 g^t_2 + \pi^{t-1}_4}{\delta}  \\ =
    - f^t_2 \frac{\pi^{t-1}_2 + \pi^{t-1}_3 g^{t-1}_1 + \delta
\pi^t_1}{\delta f^t_1}
\end{multline}
This, shifted forward two periods, simplifies to give equation
\eqref{mon:euler}.

%%% Local Variables:
%%% mode: latex
%%% TeX-master: "root"
%%% End:
