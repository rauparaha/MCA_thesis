


\chapter{Conclusions}

Time consistency is an important issue for regulators and can have serious
implications for policy effectiveness. This thesis demonstrates that
mechanism design can alleviate this problem. We construct a game that allows
the regulator to attain the first best in a setting with externalities
despite his time inconsistency.

We take a case study of a polluting monopolist and demonstrate that the
regulator's time inconsistency can adversely affect welfare. The regulator's
inability to precommit may prevent him from fully eliminating the
inefficiency of the pollution externality. The main contribution is to show
that careful policy design may provide the regulator with precommitment
opportunities. The particular mechanism considered here is a modified
version of a Pigouvian tax. Obviously, such an instrument is not appropriate
for every situation. The general implication is that
careful mechanism design could enable regulators to achieve first-best
outcomes, even in the face of obstacles such as dynamic inconsistency.

Finally, it is worth pointing out that this thesis abstracts from many
concerns facing regulators. We assume perfect information, discrete time, a
single market, a single firm serving that market and limited heterogeneity
among consumers. Further research in the field would do well to relax some
of those assumptions and investigate the impact for regulators' ability to
achieve first-best outcomes.

