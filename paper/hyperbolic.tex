\chapter[Hyperbolic discounting]{Extension: a quasi-hyperbolic discounting
model}

\label{cha:model}

In Chapter \ref{sec:dynamic-consistency} we mentioned that there are two
main causes of dynamic inconsistency: jump states and hyperbolic
discounting. The prior chapters of this thesis have dealt extensively with
the issue of regulation in the presence of jump states. We developed a
variant of a strategic delegation game to deal with inconsistency implied by
a jump state. Inefficiencies due to hyperbolic discounting can be addressed
in a similar way. In this chapter we demonstrate how the same mechanism can
also be used by a regulator who suffers from inconsistency due to
quasi-hyperbolic discounting.

We will argue that this type of inconsistency is different from the problem
posed by jump states. However, the taxation mechanism described in chapter %
\ref{cha:proposition} can still attain the socially optimal outcome.

\section{Quasi-hyperbolic discounting}

\label{sec:quasi-hyperb-disc}

Hyperbolic discounting models originated from \citet{Ainslie1992}'s
empirical work. He showed that a hyperbolic curve is a far better match for
the discount rate of most people than the standard exponential curve. When
agents use hyperbolic discounting, their intertemporal trade-offs are not
time invariant. Thus, such discount functions are not as mathematically easy
to work with as exponential functions. Because of this difficulty Ainslie's
book did not get the attention it deserved until the modifications by %
\citet{Laibson1997} which made the analysis more tractable.

Rather than adopting a full hyperbolic function, Laibson introduced an
exponential function with a modifier on the current period's discount rate.
In Laibson's model, the discount factor for the next period is $\beta \delta 
$, where $0<\beta <1$ and $0<\delta <1$. All subsequent periods are
discounted exponentially by a factor $\delta $. This modification makes the
model both simple to work with and a fair approximation of a hyperbolic
function. Time preferences with this structure are known as quasi-hyperbolic
preferences.

Quasi-hyperbolic discounting captures non-stationary time preferences, \`{a}
la hyperbolic discounting, while preserving analytical tractability.
Non-stationarity gives rise to time inconsistency, as the intertemporal
trade-off between two successive periods will change with the agent's time
reference. Consequently, precommitment has value similar to that in models
with jump states.

In the following two chapters we develop a simple quasi-hyperbolic
discounting model analogous to the monopolistic model of Chapter \ref%
{cha:problem}. Then we demonstrate the ability of the taxation game to
correct both the regulator's time inconsistency as well as the pollution
externality.

\section{The quasi-hyperbolic model}

\label{sec:quasi-hyperb-model}

As before, this model also involves a monopolist producing a good that
creates a pollution externality. The regulator addresses inefficiencies
caused by this externality through Pigouvian taxation. However, the
regulator suffers from self-control problems induced by quasi-hyperbolic
time preferences.

\section{Elements of the model}

\label{sec:elements-model}

The description of our model begins with a characterisation of the agents: a
consumer, a monopolist and a regulator.

\subsection{The consumer}

\label{sec:consumer}

Imagine a representative consumer who derives utility from two goods: a
polluting good, $x$, and a numeraire, $m$. The consumer's instantaneous
utility function is 
\begin{equation}
U(x,m)=u(x)+m.  \label{eq:27}
\end{equation}%
We assume that $u(x)$ satisfies the Inada conditions. This ensures that the
inverse demand is well-defined for all positive values of $x$.

The consumer's budget constraint is 
\begin{equation}
px+m=I.  \label{eq:28}
\end{equation}%
Therefore, inverse demand in this market is given by 
\begin{equation}
p=u^{\prime }(x).  \label{eq:31}
\end{equation}

\subsection{The monopolist}

\label{sec:monopolist}

The market for good $x$ is served by a monopolist with a cost function $C(x)$%
. His instantaneous profit function is 
\begin{align}
\pi (x)& =R(x)-C(x)  \label{eq:32} \\
& =px-C(x).
\end{align}%
Equation \eqref{eq:31} implies that profit can also be written as 
\begin{equation}
\pi (x)=u^{\prime }(x)x-C(x).  \label{eq:81}
\end{equation}

The monopolist has standard exponential time preferences and is thus time
consistent.

\subsection{The regulator}

\label{sec:regulator}

Production generates a pollution externality that is not internalised by the
monopolist. Unlike in the previous chapters, here pollution is modelled as a
stock externality rather than a flow externality. The regulator has
oversight of the monopolist and seeks to mitigate the damage wrought by the
monopolist's emissions. The stock of pollution generated by the production
of good $x$ at time $t$ is denoted by $k_{t}$.\footnote{%
For tractability in this model we have switched from flow pollution to stock
pollution.} The environmental harm caused by this stock is $\phi (k_{t})$,
where $\phi ^{\prime }(k_{t})>0$.

The stock of pollution evolves according to the following law-of-motion: 
\begin{equation}
k_{t}=\theta k^{t-1}+x^{t-1}.  \label{eq:33}
\end{equation}%
The parameter $\theta \in \lbrack 0,1]$ represents the rate of pollution
carry-over. Instantaneous welfare is thus given by 
\begin{align}
w(x_{t},k_{t})& =\pi (x_{t})+CS(x_{t})-\phi (k_{t})  \label{eq:83} \\
& =u^{\prime }(x_{t})x_{t}-C(x_{t})+\int_{0}^{x_{t}}u^{\prime
}(x)\;dx-u^{\prime }(x_{t})x_{t}-\phi (k_{t}) \\
& =u(x_{t})-C(x_{t})-\phi (k_{t}).
\end{align}

The regulator suffers from time inconsistency and is modelled as having
quasi-hyperbolic preferences. In particular, his net present valuation of
welfare from the period-$t$ perspective is 
\begin{equation}
W_{t}=w(x_{t},k_{t})+\beta \sum_{i=1}^{\infty }\delta ^{i}w(x_{t+i},k_{t+i}).
\label{eq:35}
\end{equation}%
Note the $\beta $ modifier in lifetime welfare. If $\beta =1$ then the
preferences are `exponential' and time consistent. If $\beta <1$, as is
assumed here, then preferences are `present biased' (i.e.\ non-stationary)
and the regulator will experience dynamic inconsistency.

\section{Laissez-faire equilibrium}

\label{sec:laiss-faire-equil}

First we consider the laissez-faire case. Suppose that the firm is not
regulated by the government. The monopolist does not account for the damages
arising from his pollution. As a result, the pollution stock does not appear
in his payoff. In each period, the firm solves the following static problem: 
\begin{equation}
\max_{x}u^{\prime }(x)x-C(x).  \label{eq:36}
\end{equation}

The first order condition of this problem is 
\begin{equation}
u^{\prime \prime }(x^{\ell })x^{\ell }+u^{\prime }\left( x^{\ell }\right)
-C^{\prime }\left( x^{\ell }\right) =0\text{,}  \label{eq:37}
\end{equation}%
where $x^{\ell }$ denotes the laissez-faire level of output chosen by the
monopolist. Intuitively, condition \eqref{eq:37} delivers the output level
at which marginal profit is equal to zero.

Since the monopolist discounts future payoffs exponentially, his lifetime
profit is 
\begin{equation}
\sum_{t=0}^{\infty }\delta ^{t}\pi _{t}=\frac{u^{\prime }\left( x^{\ell
}\right) x^{\ell }-C\left( x^{\ell }\right) }{1-\delta }\text{.}
\label{eq:82}
\end{equation}

\section{Benchmarking regulation}

\label{sec:benchm-regul}

\subsection{First best regulation}

\label{sec:first-best-regul}

Before we analyse the taxation game, let us first benchmark the performance
of a regulator who could choose output levels. Again, we first examine the
problem of a hypothetical regulator who can both directly determine output
and perfectly precommit to future policies. 

Suppose that the regulator can directly choose the lifetime output plan $%
\{x_{t}\}_{t=0}^{\infty }$ at time $0$. The optimal plan would solve 
\begin{multline}
W_{0}=u(x_{0})-C(x_{0})-\phi (k_{0})  \label{eq:38} \\
+\beta \sum_{t=1}^{\infty }\delta ^{t}\left[ u(x_{t})-C(x_{t})-\phi (k_{t})%
\right]
\end{multline}%
where the state variable evolves according to%
\[
k_{t}=\theta k_{t-1}+x_{t-1}, 
\]%
$\delta $ is the discount rate and $\beta $ is the quasi-hyperbolic modifier
on the future discount factor.

The regulator's optimal choice will satisfy 
\begin{equation}
w_{1}^{0}+\beta \delta \left( w_{2}^{1}-\theta w_{1}^{1}\right) =0
\label{eq:86}
\end{equation}%
in period 0, and 
\begin{equation}
w_{1}^{t}+\delta \left( w_{2}^{t}-\theta w_{1}^{t+1}\right) =0\qquad \forall
\;t\geq 2.  \label{eq:87}
\end{equation}%
for each subsequent period $t$. See Appendix \ref{sec:first-best-regul-1}
for more detail. \ 

Equations \eqref{eq:86} and \eqref{eq:87} together characterise the output
path the regulator would choose, were he able to directly control output
levels. From the perspective of the regulator at time $0$, this is the first
best output path. After substituting $w(x_{t},k_{t})$ from \eqref{eq:83}, we
obtain%
\begin{equation}
u^{\prime }(x_{t})-C^{\prime }(x_{t})-\delta \theta \left( u^{\prime
}(x_{t+1})-C^{\prime }(x_{t+1})\right) -\delta \phi ^{\prime
}(k_{t+1})=0\qquad \forall \;t\geq 2.  \label{eq:45}
\end{equation}%
This equation will be used for comparison with the laissez-faire condition,
as they have similar forms.

\subsection{Comparison to laissez-faire outcome}

\label{sec:comp-laiss-faire}

It is instructive to compare the first-best outcome to the laissez-faire
equilibrium characterised in equation \eqref{eq:37}: 
\begin{equation}
u^{\prime }\left( x^{\ell }\right) -C^{\prime }\left( x^{\ell }\right)
+u^{\prime \prime }\left( x^{\ell }\right) x^{\ell }=0.  \label{eq:46}
\end{equation}%
Remember that this condition sets the monopolist's marginal profit to zero.
Since profit is concave, marginal profit is a decreasing function.

We focus on the steady state of the model. Suppose that $x_{t}=\bar{x}%
,\forall t$, and thus $k_{t}=\bar{k},\forall t$. Now rearrange equation %
\eqref{eq:45}, 
\begin{equation}
u^{\prime }\left( \bar{x}\right) -C^{\prime }\left( \bar{x}\right) =\frac{%
\delta \phi ^{\prime }\left( \bar{k}\right) }{1-\delta \theta },
\label{eq:47}
\end{equation}%
and add $u^{\prime \prime }\left( \bar{x}\right) \bar{x}$ to both sides: 
\begin{equation}
u^{\prime }\left( \bar{x}\right) -C^{\prime }\left( \bar{x}\right)
+u^{\prime \prime }\left( \bar{x}\right) \bar{x}=\frac{\delta \phi ^{\prime
}\left( \bar{k}\right) }{1-\delta \theta }+u^{\prime \prime }\left( \bar{x}%
\right) \bar{x}.  \label{eq:48}
\end{equation}%
The left-hand side of the above equation represents the monopolist's
marginal profit evaluated at the steady-state first best output level.
Remember that marginal profit is a decreasing function. Thus, if $\frac{%
\delta \phi ^{\prime }\left( \bar{k}\right) }{1-\delta \theta }+u^{\prime
\prime }\left( \bar{x}\right) \bar{x}>0$ then $\bar{x}<x^{\ell }$ and vice
versa.

Signing the component parts gives 
\begin{align}
\delta \phi ^{\prime }\left( \bar{x}\right) & >0  \label{eq:49} \\
1-\beta \delta & >0 \\
u^{\prime \prime }\left( \bar{x}\right) \bar{x}& <0
\end{align}%
So if 
\begin{equation}
\frac{\delta \phi ^{\prime }\left( \bar{k}\right) }{1-\delta \theta }%
>u^{\prime \prime }\left( \bar{x}\right) \bar{x}  \label{eq:50}
\end{equation}%
then $\bar{x}<x^{\ell }$. The left hand side of the inequality represents
the lifetime marginal cost of the externality, while the right hand side is
the deadweight loss due to monopoly power. The externality implies that the
quantity produced may be too high from a welfare point of view, while the
firm's market power suggests that production could be too low. Taxing the
firm to reduce pollution is only worthwhile when the former effect outweighs
the latter. Henceforth, we shall assume that equation \eqref{eq:50} holds.

\subsection{Second best regulation}

\label{sec:second-best-regul}

Sophisticated regulators would recognise that they have a time inconsistency
problem. Therefore, they will try to avail themselves of a solution. If they
unable to precommit to future policies, they will act strategically to
influence the decisions of their future selves. Such behaviour would give
rise to a time-consistent second best-output path. This path would occur if
the regulator could directly choose output, but had no means of
precommitting themselves to future decisions.

To solve for the time consistent equilibrium, we must formulate the problem
recursively. Let the MPE strategy of the regulator be $x_{t}=f(k_{t})$. Then
his Bellman equation is 
\begin{equation}
U(k_{t})=\max_{x_{t}}\left\{ w(x_{t},k_{t})+\beta \delta V(\theta
k_{t}+x_{t})\right\}   \label{eq:51}
\end{equation}%
where $U(\cdot )$ is his current period's value function and $V(\cdot )$ is
his continuation value function. The continuation value function captures
the stream of future payoffs from period $t+1$ onward. It is different from
the current period's value function because quasi-hyperbolic preferences are
non-stationary. Since from next period onwards the regulator would discount
welfare exponentially, the continuation value function must satisfy the
recursive equation 
\begin{equation}
V(k_{t})=w\left( f(k_{t}),k_{t}\right) +\delta V\left( \theta
k_{t}+f(k_{t})\right) .  \label{eq:52}
\end{equation}

Dynamic programming renders a generalised Euler equation that characterises
the regulator's output strategy: 
\begin{equation}
w_{1}^{t}+\beta \delta (w_{1}^{t+1}f_{1}^{t+1}+w_{2}^{t+1})-\delta (\theta
+f_{1}^{t+1})w_{1}^{t+1}=0.  \label{eq:55}
\end{equation}

Note the difference between equation \eqref{eq:55} and equation \eqref{eq:87}%
. This difference suggests that the time consistent path will not coincide
with the first-best (i.e. precommitment) path.

\section{Regulation with delegation}

\label{sec:regul-with-deleg}

Intuitively it should be possible to solve a quasi-hyperbolic discounting
problem through `delegation' of the pricing decision.

Quasi-hyperbolic time preferences give rise to time inconsistent behaviour.
However, the internal strategic conflict between two successive regulators,
in periods $t$ and $t+1$, only concerns the choice of the period-$t+1$
action. These two regulators do not disagree about future actions, as both
will discount future payoffs exponentially. Thus, by eliminating the direct
effect of today's decision on next period's payoff, it should be possible to
render the regulator consistent.

In our delegation game, the regulator sets a tax rate for pollution
simultaneously with the monopolist's choice of output. Both the tax and the
output are feedback strategies. As in the previous chapters, we consider a
linear tax on emissions.

\subsection{The welfare function}

\label{sec:welfare-function-1}

Taxation affects the regulator's problem in two ways: first, he gains
revenue from taxation and, secondly, there is a cost to changing the tax
rate over time. Economists are often criticised that they do not account for
the cost of taxes when they recommend them. That is why we explicitly
include the costs of implementing and modifying tax schemes in the
regulator's welfare function.

Suppose that the tax is levied on emissions and the revenue from the tax is
given to consumers as a lump sum transfer. Since the marginal utility of
income to consumers is $1$, the value of the revenue in the welfare function
is equal to the cost of taxation borne by monopolist. Hence, the tax is a
simple transfer of surplus and does not change total welfare.

Period-$t$ tax revenue is $\tau _{t}x_{t}$, where $\tau _{t}$ is the tax
rate chosen by the government. Let the adjustment cost of changing policies
be $\kappa \rho (\tau _{t},\tau _{t-1})$.\footnote{%
A plausible, specific functional form might be $\left( \tau _{t}-\tau
_{t-1}\right) ^{2}$, as in Chapter \ref{sec:costs-policy-adjustm}.} Then
instantaneous welfare is given by 
\[
w(x_{t},k_{t},\tau _{t},\tau _{t-1})=u(x_{t})-C(x_{t})-\phi (k_{t})-\kappa
\rho (\tau _{t},\tau _{t-1}). 
\]%
Welfare is not directly affected by tax revenue, but changing the tax rate
over time is costly for the regulator. This assumption introduces a
`stickiness' to the tax rate. Note that if $\kappa =0$, policy adjustment
will be costless and the welfare function will not depend directly on the
tax rate.

\subsection{The profit function}

\label{sec:profit-function}

Taxation implies that the monopolist's instantaneous profit will now have
the following form: 
\begin{equation}
\pi _{t}=u^{\prime }(x_{t})x_{t}-C(x_{t})-\tau _{t}x_{t}.  \label{eq:57}
\end{equation}

\subsection{The taxation game}

\label{sec:delegation-game}

Next we set up a regulation game for this problem that mirrors the game
discussed in Chapter \ref{cha:proposition}.

\subsubsection{State variables and strategies}

\label{sec:state-variables-1}

The state variables in this game are the previous period's tax rate, $\tau
_{t-1}$, and the stock of pollution, $k_{t}$. Note that the current period's
tax rate is not a state variable, as it is set in the current period.

In each period, the monopolist chooses output simultaneously with the
regulator's choice of the current tax rate. Let the MPE strategy of the
monopolist be $x_{t}=h(\tau _{t-1},k_{t})$ and the MPE strategy of the
regulator be $\tau _{t}=g(\tau _{t-1},k_{t})$.

\subsubsection{The regulator's problem}

\label{sec:regulators-problem}

The regulator Bellman equation is now given by%
\begin{multline}
U(\tau _{t-1},k_{t})=\max_{\tau _{t}}\bigg\{w\big(h(\tau
_{t-1},k_{t}),k_{t},\tau _{t},\tau _{t-1}\big)  \label{eq:58} \\
+\beta \delta V\big(\tau _{t},\theta k_{t}+h(\tau _{t-1},k_{t})\big)\bigg\}.
\end{multline}%
The continuation value function $V$ solves the functional equation 
\begin{multline}
V(\tau _{t-1},k_{t})=w\big(h(\tau _{t-1},k_{t}),k_{t},g(\tau
_{t-1},k_{t}),\tau _{t-1}\big)  \label{eq:59} \\
+\delta V\big(g(\tau _{t-1},k_{t}),\theta k_{t}+h(\tau _{t-1},k_{t})\big).
\end{multline}

\subsubsection{The monopolist's problem}

\label{sec:monopolists-problem}

Since the monopolist discounts exponentially there is no $\beta$ in his
Bellman equation and it is standard: 
\begin{equation}
\Pi (\tau _{t-1},k_{t})=\max_{x_{t}}\left\{ \pi (x_{t},g\left( \tau
_{t-1},k_{t})\right) +\delta \Pi \left( g(\tau _{t-1},k_{t}),\theta
k_{t}+x_{t}\right) \right\} .  \label{eq:73}
\end{equation}

\subsubsection{Equilibrium strategies}

\label{sec:equil-strat-1}

From Bellman equations \eqref{eq:58}--~\eqref{eq:73} we obtain the
generalised Euler-Lagrange equations characterising the optimal strategies
for each player.

Using dynamic programming techniques, we can derive the monopolist's
Euler-Lagrange equation. It is given by%
\begin{equation}
\pi _{2}^{t}g_{1}^{t}+\frac{g_{1}^{t}}{g_{2}^{t}}\left( \theta \pi
_{1}^{t}-\pi _{2}^{t}g_{2}^{t}-\frac{1}{\delta }\pi _{1}^{t-1}\right) -\frac{%
\theta \pi _{1}^{t-1}-\pi _{2}^{t-1}g_{2}^{t-1}-\frac{1}{\delta }\pi
_{1}^{t-2}}{\delta g_{2}^{t-1}}=0.  \label{eq:84}
\end{equation}%
See Appendix \ref{sec:monopolists-problem-1} for the details.

The Euler-Lagrange equation characterising the regulator's strategy is 
\begin{multline}
\frac{w_{3}^{t-1}g_{1}^{t-1}-\beta \left(
w_{1}^{t-1}h_{1}^{t-1}+w_{3}^{t-1}g_{1}^{t-1}+w_{4}^{t-1}\right) }{\delta
h_{1}^{t-1}}-\frac{w_{3}^{t-2}}{\delta ^{2}h_{1}^{t-1}}  \label{eq:30} \\
=\beta (w_{1}^{t}h_{2}^{t}+w_{2}^{t}+w_{3}^{t}g_{2}^{t})-g_{2}^{t}w_{3}^{t}
\\
+\frac{\left( \theta +h_{2}^{t}\right) }{h_{1}^{t}}\left[
w_{3}^{t}g_{1}^{t}-\beta \left(
w_{1}^{t}h_{1}^{t}+w_{3}^{t}g_{1}^{t}+w_{4}^{t}\right) -\frac{w_{3}^{t-1}}{%
\delta }\right] .
\end{multline}%
The derivations are detailed in Appendix \ref{sec:regulators-problem-1}.

To compare this game to the first-best outcome, let us consider the special
case where $\kappa =0$, so policy change is costless. We substitute in the
following partial derivatives of the welfare function: 
\begin{align}
w_{1}^{t}& =u^{\prime }(x_{t})-c^{\prime }(x_{t})  \label{eq:70} \\
w_{2}^{t}& =-\phi ^{\prime }(x_{t}) \\
w_{3}^{t}& =0 \\
w_{4}^{t}& =0.
\end{align}%
After this substitution, Euler equation \eqref{eq:30} reduces to 
\begin{equation}
\frac{u^{\prime }(x_{t-1})-c^{\prime }(x_{t-1})}{\delta }-\phi ^{\prime
}(x_{t})-\theta \big(u^{\prime }(x_{t})-c^{\prime }(x_{t})\big)=0.
\label{eq:71}
\end{equation}%
Multiplying by $\delta $ and shifting the equation forward one period yields 
\begin{equation}
u^{\prime }(x_{t})-c^{\prime }(x_{t})-\delta \phi ^{\prime }(x_{t+1})-\delta
\theta \big(u^{\prime }(x_{t+1})-c^{\prime }(x_{t+1})\big)=0\text{.}
\label{eq:72}
\end{equation}%
Note that this replicates Euler equation \eqref{eq:45} that characterizes
the precommitment outcome. Therefore, the regulation game will deliver the
first-best output path.