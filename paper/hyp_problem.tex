
\chapter{A quasi-hyperbolic discounting model}
\label{cha:model}

It was mentioned in Chapter \ref{sec:dynamic-consistency} that there
are two main causes of dynamic inconsistency: jump states and
hyperbolic discounting. The main Part of this thesis has dealt
extensively with the issue of regulation when jump states exist. It
developed a variant of a strategic delegation game to overcome the
inconsistency implied by the jump state. However, there is a subset of
the hyperbolic discounting problem that can also be dealt with in the
same fashion. In this Part we briefly outline and demonstrate how the
same mechanism can also be used by a regulator who suffers from
inconsistency due to quasi-hyperbolic discounting.

It will be shown that the problem created is different from the
problem posed by jump states; however, the taxation mechanism
described in chapter \ref{cha:proposition} can still overcome both of
the regulator's problems.

\section{Quasi-hyperbolic discounting}
\label{sec:quasi-hyperb-disc}

The root of hyperbolic discounting models is in \citet{Ainslie1992}'s
empirical work that showed a hyperbolic curve is a far better match
for the discount rate of most people than the standard exponential
curve. However, hyperbolic discount functions are not as mathematically
easy to work with as exponential functions since the discount rate is
not constant. That initially hindered the impact of Ainslie's book
until \citet{Laibson1997} modified the functional form slightly.

Rather than implementing a full hyperbolic function, Laibson
introduced an exponential function with a modifier on the first
period's discount rate. For the first period he set the discount rate
at $\beta\delta$, where $\delta$ is the usual rate of time preference
and $0 < \beta < 1$ is the modifier. For every period after that the
discount rate in Laibson's model is the usual $\delta$, which makes
the model both simple to work with and a fair approximation of a
hyperbolic function. This canonical paper introduced the idea of what
is now known as quasi-hyperbolic discounting.

Quasi-hyperbolic discount rates have the advantage of creating
non-stationary preferences, \`{a} la hyperbolic discounting, while
still being simple to work with. The non-stationarity of preferences
also creates time inconsistency since the discount rate for period
$t+n, \forall n>1$ is not the same now as it will be in period $t+n-1$
when the agent chooses $t+n$'s action. Consequently, precommitment
has value similar to the discussion of jump states earlier.

What is notable for our purposes is that quasi-hyperbolic discounting
causes time inconsistency by creating different preferences across
periods $t$ and $t+1$. That implies that a mechanism that can create
`precommitment' across those periods can generate time consistent
behaviour and our regulatory game is just such a mechanism.

In the following two chapters we develop a simple quasi-hyperbolic
discounting model, analagous to the monopolistic model of Part
\ref{part:jump}, and demonstrate the efficacy of the regulation game
to overcome both the regulator's time inconsistency and the pollution
externality.

\section{The quasi-hyperbolic model}
\label{sec:quasi-hyperb-model}

The model we construct here also involves a monopolist producing a
good that creates a pollution externality. The regulator has to
internalise the externality through Pigouvian taxation. Unfortunately
for the regulator they suffer from time inconsistency generated via a
quasi-hyperbolic welfare function.

\section{Elements of the model}
\label{sec:elements-model}


A description of the model begins by characterising the agents: a
consumer, a monopolist and a regulator.

\subsection{The consumer}
\label{sec:consumer}

Imagine a representative consumer with quasi-linear instantaneous
preferences across two goods: a polluting good, $x$, and a numeraire,
$m$. The consumer's instantaneous utility function is
\begin{equation}
  \label{eq:27}
  U(x,m) = u(x) + m.
\end{equation}
The function $u(x)$ must satisfy the Inada conditions in order for the
inverse demand to be defined over all possible values of
$x$.

The consumer's budget constraint is
\begin{equation}
  \label{eq:28} p x + m = I,
\end{equation} hence, inverse demand in this market is
\begin{equation}
  \label{eq:31}
  p=u'(x).
\end{equation}

\subsection{The monopolist}
\label{sec:monopolist}

The manufacturer of good $x$ has a cost function, $C(x)$, and
monopolises the market. His instantaneous profit function is
\begin{align}
  \label{eq:32}\pi &= R(x) - C(x) \\
  &= px - C(x).
\end{align}
Remembering equation \eqref{eq:31} allows us to simplify this to
\begin{equation}
  \label{eq:81}
  \pi = u'(x)x - C(x).
\end{equation}

The monopolist discounts in the standard exponential fashion and is
thus time consistent.

\subsection{The regulator}
\label{sec:regulator}

The monopolist's production generates pollution which is not accounted
for by the producer. Pollution is modelled as a stock externality,
rather than a flow externality. The regulator has oversight of the
monopolist and seeks to mitigate the damage wrought by the
monopolist's emissions.  The stock of pollution generated by the
production of good $x$ at time $t$ is $k_t$.\footnote{For tractability
  in this model we have switched from flow pollution to stock
  pollution.} The environmental harm caused by this stock is
$\phi(k_t)$, an increasing function of $k_t$.

The stock of the good evolves according to the rule
\begin{equation}
  \label{eq:33} k_t = \theta k^{t-1} + x^{t-1} \qquad \theta \in [0,1].
\end{equation} Instantaneous welfare is given by
\begin{align}
  \label{eq:34}w(x_t,k_t) &= \pi(x_t) + CS(x_t) - \phi(k_t) \\
  &=  u'(x_t).x_t - C(x_t) + \int_0^{x_t} u'(x)\; dx - u'(x_t).x_t -
  \phi(k_t) \\ \label{eq:83}
  &= u(x_t) - C(x_t) - \phi(k_t).
\end{align}

The regulator suffers from time inconsistency and is modelled as
having quasi-hyperbolic preferences with $\beta\delta$-discounting. As
a result, the regulator's net present valuation of welfare is
\begin{equation}
  \label{eq:35}
  W(x,k) = w(x_t,k_t) + \beta \sum_{i=1}^\infty \delta^i w(x_{t+i},k_{t+i}).
\end{equation}
Note the $\beta$ modifier on the future payoffs. If $\beta=1$ then the
preferences are `normal' and time consistent. If $\beta<1$, as is
assumed here, then preferences are non-stationary and the regulator
will experience dynamic inconsistency.

\section{Laissez-faire equilibrium}
\label{sec:laiss-faire-equil}

First consider the case in which the monopolist acts
unregulated. Damage caused by emissions is not internalised by the
monopolist. As a result, there is no stock of pollution to consider
when considering the monopolist's decision and so the problem is not
truly dynamic. The monopolist simply solves a static profit
maximisation problem in each period:
\begin{equation}
  \label{eq:36}
  \max_x u'(x).x - C(x).
\end{equation}

The first order condition of the problem defined by equation
\eqref{eq:36} is
\begin{equation}
  \label{eq:37}
  u''\left( x^{\ell}\right) .x^{\ell} + u'\left( x^{\ell}\right)  - C'\left( x^{\ell}\right)  = 0
\end{equation}
where $x^{\ell}$ denotes the laissez-faire level of output chosen by
the monopolist. Equation \eqref{eq:37} indicates the point at which
marginal profit is equal to zero.

Since the monopolist discounts in the standard, exponential fashion
his lifetime profit is
\begin{equation}
  \label{eq:82}
  \sum_0^\infty \pi_t = \frac{u'\left(x^{\ell}\right) x^{\ell} -
    C\left( x^{\ell}\right)}{1-\delta}
\end{equation}


\section{Benchmarking regulation}
\label{sec:benchm-regul}

\subsubsection{First best regulation}
\label{sec:first-best-regul}

Before exploring the taxation game let us first turn to, once again,
benchmarking the regulator's performance. As before, we first examine
the problem that would face a hypothetical regulator, who can both
directly determine output and perfectly precommit to future output.

Suppose that the regulator can directly choose $\{ x_t
\}_{t=0}^\infty$ at time $0$. The regulator must maximise
\begin{multline}
  \label{eq:38}W_0(x,k) = u(x_0) -
  C(x_0)  - \phi(k_0)\\ + \beta \sum_{t=1}^\infty \delta^i \left[ u(x_t) -
    C(x_t) - \phi(k_t) \right]
\end{multline}
where $k_t = \theta k_{t-1} + x_{t-1}$ describes the evolution of the
state variable, $\delta$ is the discount rate and $\beta$ is the
quasi-hyperbolic modifier on the future discount rate.

The regulator's optimal strategy must satisfy
\begin{equation}
  \label{eq:86} w^0_1 + \beta\delta \left(w^1_2 - \theta w^1_1 \right)
  = 0,
\end{equation}
in the first period, and
\begin{equation}
  \label{eq:87} w^t_1 + \delta \left(w^t_2 - \theta w^{t+1}_1 \right)
  = 0 \qquad \forall\; t \geq 2.
\end{equation}
for each period after that. See Appendix \ref{sec:first-best-regul-1}
for a derivation of the result.

Equations \eqref{eq:86} and \eqref{eq:87} together characterise the
output path that the regulator would choose, were he able to directly
control output levels. From the perspective of the regulator at time
$0$, this is the first best output path.

After substituting $w(x_t,k_t)$ from equation \eqref{eq:83} this
expands to
\begin{equation}
  \label{eq:45}
  u'(x_t) - C'(x_t) - \delta\theta \left( u'(x_{t+1}) -
    C'(x_{t+1})\right) - \delta\phi'(k_{t+1}) = 0 \qquad \forall\; t
  \geq 2.
\end{equation}
This form will be helpful for comparison since it splits out the
components of the equation.


\subsubsection{Comparison to laissez-faire outcome}
\label{sec:comp-laiss-faire}

At this point it is important to see how the first-best outcome
compares to the laissez-faire outcome previously derived. The laissez
faire equilibrium is characterised by equation \eqref{eq:37}:
\begin{equation}
  \label{eq:46}
  u'\left( x^{\ell}\right) - C'\left( x^{\ell}\right) + u''\left(
    x^{\ell}\right) x^{\ell} = 0.
\end{equation}
This sets the monopolist's marginal profit to zero. Since profit is
concave, marginal profit is a decreasing function.

In order to tractably compare this equilibrium condition to the first
best outcome it is necessary to focus upon the steady state of the
dynamic game. Suppose that $x_1 = x_2 = \ldots = x_T = \bar{x}$, and
similarly for the state variable, $k_t$. Now rearrange equation
\eqref{eq:45},
\begin{equation}
  \label{eq:47}
  u'\left( \bar{x}\right) - C'\left( \bar{x}\right) =
  \frac{\delta\phi'\left( \bar{k}\right)}{1-\delta\theta},
\end{equation}
and add $u''\left( \bar{x}\right) \bar{x}$ to both sides:
\begin{equation}
  \label{eq:48}
  u'\left( \bar{x}\right) - C'\left( \bar{x}\right) +
  u''\left( \bar{x}\right) \bar{x} = \frac{\delta\phi'\left(
      \bar{k}\right)}{1-\delta\theta} + u''\left( \bar{x}\right)
  \bar{x}.
\end{equation}
The LHS of this equation represents the monopolist's marginal profit
function evaluated at the steady state, first best output
level. Remembering that marginal profit is a decreasing function, if
$\frac{\delta\phi'\left( \bar{k}\right)}{1-\delta\theta} + u''\left(
  \bar{x}\right) \bar{x} > 0$ then $\bar{x} < x^{\ell}$ and vice
versa.

Signing the component parts gives
\begin{align}
  \label{eq:49}
  \delta\phi'\left( \bar{x}\right) &> 0 \\
  1-\beta\delta &> 0 \\
  \intertext{by assumption, and}
  u''\left( \bar{x}\right) \bar{x} &< 0
\end{align}
because $u(\cdot)$ exhibits diminishing marginal utility. So if
\begin{equation}
  \label{eq:50}
  \frac{\delta\phi'\left( \bar{k}\right)}{1-\delta\theta} > u''\left(
    \bar{x}\right) \bar{x}
\end{equation}
then $\bar{x} < x^{\ell}$. The left hand component of the inequality
shows the lifetime marginal cost of the externality, while the right
hand component shows the deadweight loss due to the exercise of
monopoly power. The externality effect causes the quantity produced to
be too high, while the firm's market power depresses
production. Regulation to diminish pollution is only worthwhile when
the former outweighs the latter. Henceforth, we shall assume that
equation \eqref{eq:50} holds.

\subsubsection{Second best regulation}
\label{sec:second-best-regul}

Most regulators will be smart enough to realise that they have a time
inconsistency problem and to try to avail themselves of a
solution. If they are sophisticated then, in the absence of
precommitment, the solution is to act in such a way now that there
is no incentive to deviate in the future. This is the time consistent,
second best output path. It is the one that a regulator would pick if
they could directly regulate output, realised that they were time
inconsistent and had no means of precommitting themselves to future
decisions.

The problem must be formulated recursively in order to solve for a
time consistent output path. Let the MPE strategy of the regulator be
$x_t = f(k_t)$. Then his Bellman equation is
\begin{equation}
  \label{eq:51}
  U(k_t) = \max_{x_t} \left\{ w(x_t,k_t)
    + \beta\delta V( \theta k_t + x_t) \right\}
\end{equation}
where $U(\cdot)$ is his current period's value function and $V(\cdot)$
is his continuation value function. The continuation value function
describes the stream of future payoffs from period $t+1$ onward. It is
different from the current period's value function because
quasi-hyperbolic preferences are non-stationary.
\begin{equation}
  \label{eq:52}
  V(k_t) = w\left( f(k_t),k_t\right) +
  \delta V\left( \theta k_t + f(k_t) \right).
\end{equation}

Solving for the regulator's generalised Euler equation gives
\begin{equation}
  \label{eq:55}
  w^t_1 + \beta\delta( w^{t+1}_1 f^{t+1}_1
  + w^{t+1}_2 ) - \delta (\theta + f^{t+1}_1)w^{t+1}_1 = 0.
\end{equation}

Note the differences between equation \eqref{eq:55} and equation
\eqref{eq:87}. That difference characterises the cost to the regulator
of having to overcome their own time inconsistency without outside
assistance.




%%% Local Variables:
%%% mode: latex
%%% TeX-master: "root"
%%% End:

% LocalWords:  externality precommitment
