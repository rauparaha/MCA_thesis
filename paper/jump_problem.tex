\chapter{Problem}

\label{cha:problem} % \textit{To do in this section:}
%   \begin{itemize}
%     \item \textit{Point out that inconsistency in the profit function
%       of the firm doesn't necessarily translate into inconsistency in
%       the regulator's welfare function. It may be that the
%       inconsistency is in the demand function and just appears in the
%       welfare function as a transfer between consumers and
%       firms. However, in the case of the durable goods monopoly, the
%       consumer surplus function is also a function of the jump
%       variable and so the regulator's welfare function is certain to
%       be inconsistent.}
%   \item \textit{Add a section talking about the Coase Conjecture and
%       why it doesn't apply here.}
%     \end{itemize}

The problem of regulating a dynamically inconsistent polluter is
complicated: a regulator seeking to maximise social welfare will have to
take into account the dynamic structure of consumer demand for the
polluter's product. This implies that instantaneous welfare will contain a
jump variable and hence the polluter's time consistency problem will be
transferred to the regulator. In the absence of precommitment devices a
regulator faced with a dynamically inconsistent polluter may be unable to
attain the first-best outcome. A likely consequence of this failure is
over-pollution.

This chapter presents a model of a polluting durable goods monopoly and
shows how its regulation may be subject to time inconsistency. We will
characterise both the first-best (precommitment) and the second-best
(time-consistent) regulatory outcomes. They will then be used as benchmarks
for the proposed taxation mechanism in the next chapter.

\section{A model of a dynamically inconsistent regulator}

\label{sec:model-dynam-incons} In each situation described in section \ref%
{sec:industr-which-dynam}, a regulator will be affected by the dynamic
inconsistency of the regulated firm. In this chapter we set up a model to
illustrate the problem. We will focus on the specific case of a durable
goods producer, which will serve to illustrate how our proposal can overcome
the regulatory problems induced by jump states.

In this section we elucidate the model and describe the players' decisions
and interactions. The main decision makers are a regulator and a monopolist
who supplies an infinitely durable good to a mass of consumers. We consider
a durable goods producer because their behaviour is known to produce dynamic
inconsistency (see section \ref{sec:durable-goods}).

\subsection{The consumption decision}

\label{sec:consumption-decision}

The monopolist produces an infinitely durable good, which is then supplied
to a mass of consumers. For simplicity this mass is normalised to $1$. The
period-$t$\ price is denoted by $p^{t}$. Each consumer can buy only one unit
of the good in their lifetime. \ After the purchase they withdraw from the
market, but continue to enjoy a stream of benefits $v\in \lbrack 0,1]$\ in
perpetuity. Future benefits are discounted by a factor $\beta $.\footnote{%
The parameter $\beta $  can also capture depreciation of the benefits of the
durable good.}

The taste parameter, $v$, indicates an individual consumer's valuation of
the good. Consumers are assumed to have heterogeneous valuations and their
tastes are distributed across the population according to a probability
density function $\phi (v)$, which has a corresponding cumulative density
function (CDF) denoted by $\Phi (v)$.

\subsubsection{Price trajectory}

\label{sec:price-trajectory}

Consumers decide the timing of their purchases. Those who are most eager
will be willing to pay a premium to buy the good early on. Hence, initially
the price of the good will be high. As the early adopters leave the market
the price of the good will be reduced in order to entice the remaining
consumers to make a purchase. Consequently, the equilibrium price path will
be decreasing over time.

It must be noted that it may not be optimal for the monopolist to operate in
all periods. As Stokey originally showed, a monopolist with constant
marginal costs and precommitment power will choose to produce only in the
first period \citep{Coase1972, Stokey1981}. Two key features of the model
studied here distinguish it from Stokey's:

\begin{enumerate}
\item the monopolist does not have precommitment power; and,

\item the monopolist has convex costs (assumption \ref{dur:ass2}).
\end{enumerate}

As shown by \citet{Kahn1986}, the presence of either of these properties
invalidates Stokey's result. Convexity of costs implies that higher
production will increase marginal costs, and thus it induces the monopolist
to smooth production over time costs. Kahn shows that, in an infinite
horizon setting, the monopolist's production smoothing will cause them to
asymptotically approach a steady-state price. Thus, the good's price
sequence is Cauchy and converges to a limit (i.e. a steady state price) as $%
t\rightarrow \infty $: 
\begin{equation}
p^{t-1}-p^{t}\geq p^{t}-p^{t+1}\quad \forall \;t  \label{eq:1}
\end{equation}%
This\ dynamic stability condition constrains the price trajectory in the
market under consideration.

\subsubsection{Consumers' intertemporal trade-offs}

\label{sec:intert-trade-offs}

In a durable goods market the consumer's main decision is when to purchase.
Knowing that the price will decline over time, they weigh the benefit of
purchasing in the current period against the expected price reduction of the
subsequent period. If the expected cost reduction from waiting until the
next period outweighs the foregone $v$, they will delay purchase.

The lifetime net benefit, $V^{t}$, of purchasing in period-$t$ is the net
present value of the stream of discounted benefits, minus the cost of
purchase: 
\begin{equation}
V^{t}=\frac{v}{1-\beta }-p^{t}.  \label{eq:2}
\end{equation}
Similarly, 
\begin{equation}  \label{eq:3}
V^{t+1}=\frac{\beta v}{1-\beta }-\beta p_{e}^{t+1},
\end{equation}
where $p_{e}^{t+1}$ is the anticipated future price. Thus, purchase will be
delayed in period-$t$ if 
\begin{gather}
V^{t+1}>V^{t}  \label{eq:4} \\
v<p^{t}-\beta p_{e}^{t+1}.
\end{gather}

By shifting equations \eqref{eq:3} and \eqref{eq:2} back one period and
performing a similar rearrangement, one finds that purchase will be delayed
in period-$t-1$ if 
\begin{equation}
v<p^{t-1}-\beta p^{t}.  \label{eq:5}
\end{equation}

Suppose that consumers have rational expectations: $p_{e}^{t+1}=p^{t}$. Then
a consumer who chooses not to delay purchase in the current period will
never prefer to consume in a later period. This can be seen by combining
equation \eqref{eq:1} with $\beta <1$ and shifting it forward one period to
give $p^{t}-p^{t+1}>\beta (p^{t+1}-p^{t+2})$, which rearranges to 
\begin{equation}
p^{t}-\beta p^{t+1}>p^{t+1}-\beta p^{t+2}.  \label{eq:6}
\end{equation}%
Equation \eqref{eq:4} implies that all consumers who choose to buy in the
current period have tastes such that $v\geq p^{t}-\beta p_{e}^{t+1}$. Hence, 
\begin{equation}
v\geq p^{t}-\beta p^{t+1}>p^{t+1}-\beta p^{t+2}>\ldots >p^{T-1}-\beta
p^{T},\quad \forall T>t,  \label{eq:7}
\end{equation}%
and each consumer who buys the good in period-$t$ will be worse off delaying
the purchase to any future period.

\subsubsection{The demand function}

\label{sec:demand-function}

The above discussion of consumer behaviour enables us to obtain the demand
function for a given period $t$. Current demand is the mass of people who
chose to delay purchase in the previous period but choose not to delay from
the current period to the next. Combining the equations describing consumer
choice with the CDF of $v$ gives us the mass of consumers who purchase in
period $t$. The resulting demand is 
\begin{equation}
x^{t}=x(p^{t-1},p^{t},p_{e}^{t})=\Biggl\{%
\begin{array}{lcl}
\Phi (p^{t-1}-\beta p^{t})-\Phi (p^{t}-\beta p_{e}^{t+1}) & \mbox{if} & t>0,
\\ 
1-\Phi (p^{0}-\beta p^{1}) & \mbox{if} & t=0.%
\end{array}
\label{eq:demand}
\end{equation}%
Since the function $\Phi (\cdot )$\ is a CDF, it must increasing. Equation %
\eqref{eq:6} implies that $\Phi (p^{t-1}-\beta p^{t})>\Phi (p^{t}-\beta
p_{e}^{t+1})$, so demand will be positive in all periods. Furthermore, we
need the demand function to be concave in both the current and future
prices. This, along with Assumption \ref{dur:ass2}, will guarantee the
concavity of profits.

\begin{ass} \label{dur:ass1}
  The demand function function, $x^t$,
  satisfies $\ddel{x^t}{(p^t)} \leq 0$, $\ddel{x^t}{(p^{t+1}_e)} \leq
  0$.
\end{ass}

\subsection{The production decision}

\label{sec:production-decision}

Suppose that the monopolist incurs operating costs $C(x^{t})$. Thus, his
profit function is 
\begin{equation}
\pi ^{t}=\pi
(p^{t-1},p^{t},p_{e}^{t+1})=p^{t}x(p^{t-1},p^{t},p_{e}^{t+1})-C\left(
x(p^{t-1},p^{t},p_{e}^{t+1})\right) .  \label{dur:profit}
\end{equation}%
To ensure concavity of instantaneous profits, it is necessary to assume
convexity of the cost function. As discussed above, this also implies that a
monopolist who is able to precommit will not cease production after the
first period \citep{Kahn1986}. 
\begin{ass} \label{dur:ass2}
The monopolist's cost function satisfies  $C''(x^t) > 0$.
\end{ass}Along with Assumption \ref{dur:ass1} this assumption guarantees
that profits will be concave in both the present price and next period's
price: $\ddel{\pi^t}{(p^t)}\leq 0$, $\ddel{\pi^t}{(p^{t+1}_e)}\leq 0$. See
Appendix \ref{cha:Assumpproof} for a proof.

\subsection{The regulator's decision}

\label{sec:regulators-decision}

We assume that the regulator seeks to maximise social welfare. They take in
to account the monopolist's profit, the consumer surplus, and potential
externalities arising from production. In the partial equilibrium model
studied here, the consumer surplus considered is solely the net benefit
gained by consumers from purchasing the monopolist's good. The partial
equilibrium nature of this model makes it difficult to speak accurately of
welfare but we use it as a convenient approximation, having noted that this
is not a true welfare analysis. Consumer surplus is given by 
\begin{align}
CS^{t}& =CS(p^{t-1},p^{t},p_{e}^{t+1}) \\
& =\int_{p^{t}-\beta p_{e}^{t+1}}^{p^{t-1}-\beta p^{t}}\phi
(v)(v-p^{t})\;dv+\int_{p^{t-1}-\beta p^{t}}^{1}\phi (v)v\;dv  \label{eq:8} \\
& =E(v-p^{t}|p^{t}-\beta p_{e}^{t+1}\leq v<p^{t-1}-\beta p^{t})+E(v|v\geq
p^{t-1}-\beta p^{t}).
\end{align}

In addition to consumer surplus and profits, the regulator must also
consider the environmental impact of pollution generated by production. \
Our setting assumes that pollution is a flow, rather than a stock,
externality. That is, instead of modelling pollution as a stock of harm that
accumulates over time, we consider harm that is caused by the polluter's
current production. In other words, we assume that the level of emissions is
an increasing function, $\psi (x^{t})$, of current output. Specifically, the
pollution function satisfies the following condition. 
\begin{ass} \label{dur:ass3} The pollution function satisfies
  $\psi'(x^t) > 0$ and has the initial condition $\psi(0) = 0$.
\end{ass}

Such an assumption is plausible for some, but not all, types of pollution.
For example, air pollution is quickly dispersed and may have little time to
accumulate. Thus, it can be reasonably modelled as a flow externality. Heavy
metals, however, can accumulate in the soil and eventually reach harmful
levels. It would be more suitable to consider them as stock pollutants.
Either approach is valid for a subset of pollution problems. We have chosen
flow pollution because the framework facilitates simple analysis of time
inconsistent regulation.

Our assumptions yield the following instantaneous welfare function: 
\begin{align}
w^{t}& =w(p^{t-1},p^{t},p_{e}^{t+1})  \label{dur:welfare1} \\
& =CS(p^{t-1},p^{t},p_{e}^{t+1})+\pi (p^{t-1},p^{t},p_{e}^{t+1})-\psi \left(
x(p^{t-1},p^{t},p_{e}^{t+1})\right) \\
& =\int_{p^{t}-\beta p_{e}^{t+1}}^{1}\phi
(v)v\;dv-C(x(p^{t-1},p^{t},p_{e}^{t+1}))-\psi (x(p^{t-1},p^{t},p_{e}^{t+1})).
\end{align}
Given rational expectations ($p_{e}^{t+1}=p^{t}$), the above definition
implies the presence of a jump variable in the regulator's objective
function. As argued in section \ref{sec:suff-cond-time} this may cause a
time consistency problem for the regulator.

\section{Laissez-faire performance}

\label{sec:laiss-faire-perf}

Having described the agents in the market and their interactions, we now
turn to the topic of efficiency. It could be that this market is already
efficient. It is also possible that the regulator may not be able to improve
on the free market outcome. The first step in assessing any government
intervention is to characterise the performance of the unregulated market.
Then in section \ref{sec:second-best-price} we will compare the laissez
faire outcome to the regulator's benchmark pricing policies.

\subsection{The monopolist's price path}

\label{sec:monop-price-path}

As already established, the monopolist's demand function is 
\begin{equation}
x^{t}=\Phi (p^{t-1}-\beta p^{t})-\Phi (p^{t}-\beta p_{e}^{t+1})\text{.}
\end{equation}%
It generates the following instantaneous profit function: 
\begin{multline}
\pi (p^{t-1},p^{t},p_{e}^{t+1})=p^{t}\left[ \Phi (p^{t-1}-\beta p^{t})-\Phi
(p^{t}-\beta p_{e}^{t+1})\right]  \\
-C\left( \Phi (p^{t-1}-\beta p^{t})-\Phi (p^{t}-\beta p_{e}^{t+1})\right) .
\label{eq:9}
\end{multline}%
The monopolist's objective is maximization of lifetime profits. Suppose that
his discount factor is $\delta $. To get an interior equilibrium, we assume
that $\beta <\delta <1$.

The presence of a jump state, $p_{e}^{t+1}$, suggests that profit
maximization will give rise to a time inconsistency problem for the
monopolist. This problem can be conceptualised as a strategic conflict
between the monopolist's current and future `selves'. We will model the time
consistent sequence of decisions as the subgame perfect equilibrium outcome
of a dynamic `intrapersonal' game. This game is played by different agents,
each a temporal incarnation of the monopolist associated with a particular
time period. These temporal `selves' choose prices to maximise their payoffs
while accounting for the discrepancy between their interests and those of
future agents. In that our agents are `sophisticated' in the sense of %
\citet{Donoghue2001}.

Our analysis will focus on a particular type of subgame perfect equilibrium.
Namely, we consider the Markov-perfect equilibrium of the monopolist's
intrapersonal game. This solution concept restricts the players' strategies
to be functions of the current state: $p^{t}=g(p^{t-1})$. The Markovian
approach allows us to use dynamic programming techniques to characterise the
equilibrium. It assumes away history-dependent punishments with trigger
strategies.

Furthermore, we need to specify how current agents form expectations about
the behaviour of future players. \citet{Stokey1981} shows that rational
expectations are not sufficient to prevent a multiplicity of equilibria
similar to that of the Folk Theorem. To avoid this multiplicity, we assume
that decision makers have perfect rational expectations. That is, we require
expectations of future prices to be correct both on and off the equilibrium
path. The restriction to Markovian strategies and perfect rational
expectations imply that $p_{e}^{t+1}=g(p^{t})$.

If the monopolist does not have precommitment power and follows a Markov
strategy $g(p)$, then his optimal decision rule must solve the Bellman
equation 
\begin{equation}
\Pi (p^{t-1})=\max_{p^{t}}\Bigl\{\pi \bigl(p^{t-1},p^{t},g(p^{t})\bigr)%
+\delta \Pi (p^{t})\Bigr\}.  \label{eq:10}
\end{equation}%
This equation recursively defines each player's lifetime payoff, as captured
by the value function $\Pi (p)$,\ in terms of future players' lifetime
payoffs. That is, the current player's equilibrium lifetime payoff is\ the
maximised value of his instantaneous payoff from his decision, plus the
anticipated continuation payoff of tomorrow's decision. Furthermore, the
optimal pricing strategy is time invariant, implying that 
\begin{equation}
g(p^{t-1})=\arg \max_{p^{t}}\Bigl\{\pi \bigl(p^{t-1},p^{t},g(p^{t})\bigr)%
+\delta \Pi (p^{t})\Bigr\}.  \label{eq:11}
\end{equation}

In Appendix \ref{cha:equil-durable-goods} we use dynamic programming
techniques to characterise the Markov-perfect equilibrium. Differentiation
yields a first-order condition and an envelope condition. Combining these
delivers a generalised Euler-Lagrange equation 
\begin{equation}
\pi _{2}^{t}+\pi _{3}^{t}g_{1}^{t+1}+\delta \pi _{1}^{t+1}=0,  \label{eq:12}
\end{equation}%
where the subscript $i$ denotes the partial derivative with respect to the
function's $i$-th argument.\footnote{%
For example, $\pi _{2}^{t}=\partial \pi (p^{t-1},p^{t},p^{t+1})/\partial
p^{t}$.} This difference-differential equation implicitly characterises the
price path that will be chosen in equilibrium. The first term of this
equation reflects the direct effect of $p^{t}$ on the profit function. The
second term encapsulates the effect of $p^{t}$ on $\pi ^{t}$ via the
strategy choice, $g(p^{t})$, of the subsequent agent. The final term
captures the discounted effect of the current action on future profits.
Optimality requires that these effects sum to 0.

\subsection{Inefficiency of the laissez-faire outcome}

\label{sec:need-regulation}

Next we investigate whether the laissez-faire price path derived in section %
\ref{sec:monop-price-path} provides a rationale for government intervention.
The definitive answer to this question requires characterising the
regulatory equilibrium. Intuitively, our setting exhibits a number of
inefficiencies which suggest that regulation may improve welfare. First, the
market is served by a monopolist. When a producer has market power, he
usually tends produce too little and charge a price above the efficient
level. Second, production causes a negative externality as pollution is
released. Such externalities may imply overproduction relative to the
first-best outcome. Finally, the monopolist's profit function contains a
jump variable. Section \ref{sec:suff-cond-time} showed how this may create a
time consistency problem for the producer. This problem may induce the
monopolist to increase production, further exacerbating potential
overprovision of the durable good.

The three welfare effects mentioned above act in different directions.
Monopoly power pushes the price above efficiency. Conversely, pollution and
the monopolist's time consistency problem tend to reduce the price below the
efficient level. We cannot predict the net effect without knowledge of
market specifics. It may even be possible for the three inefficiencies to
exactly cancel each other out. However, as section \ref%
{sec:second-best-price} discusses, it is unlikely that the laissez-faire
price will replicate the efficient path. Since we do not exclude the
possibility of a negative tax (i.e. a subsidy), there is no need to assume
that a particular type of inefficiency is dominant. The proposal described
in Chapter \ref{cha:proposition} will still yield the first-best outcome.

\section{Benchmarking regulatory performance}

\label{sec:benchm-regul-perf}

To evaluate the regulatory intervention proposed in Chapter \ref%
{cha:proposition}, we first need to study the price paths attainable by a
regulator. For the purpose of benchmarking we assume that in each period the
regulator has direct control the price of the durable good. Such equilibria
would arise in standard models of Pigouvian taxation \citep{Benchekroun1997}.

This section characterises two benchmark plans,

\begin{enumerate}
\item the first-best price path where the regulator can commit to  future
prices; and,

\item the second-best price path where the regulator is unable to  commit to
future prices.
\end{enumerate}

\subsection{First-best price plan}

\label{sec:first-best-price}

\subsubsection{Deriving the plan}

\label{sec:deriving-path}

The regulator's first-best price path is the plan he would choose in the
first period if at that time he could precommit to a complete sequence of
future prices. Given rational expectations, ($p_{e}^{t+1}=p^{t+1}$), this
plan can be obtained as the open-loop Nash equilibrium (OLNE) of the problem 
\begin{equation}
\max_{\vect{p}^{t}}\sum_{t=1}^{\infty }\delta
^{t-1}w^{t}(p^{t-1},p^{t},p^{t+1})\text{,}  \label{eq:13}
\end{equation}%
where $\vect{p}^{t}$ is the price vector $\vect{p}^{t}=\{p^{t}\}_{t=1}^{%
\infty }$. The solution to this problem is a sequence of prices, indexed by
time, that maximises the net present value of total welfare.

The optimal precommitment price path, $\vect{p}^{t}$, satisfies the
first-order conditions 
\begin{align}
w_{2}^{t}+\delta w_{1}^{t+1}& =0,\quad t=1  \label{four} \\
w_{3}^{t-1}+\delta w_{2}^{t}+\delta ^{2}w_{1}^{t+1}& =0,\quad t\geq 2.
\label{five}
\end{align}%
The above equations are obtained by differentiating period-1 lifetime
welfare with respect to the period-1 price $p^{1}$\ and an arbitrary future
price $p^{t},$\ $t>1$. Since these prices solve an unconstrained
maximisation problem, they attain the highest possible net present value of
welfare.

Substituting the derivatives of instantaneous welfare \eqref{dur:welfare1}
into \eqref{four} and \eqref{five} yields equivalent conditions expressed in
terms of monopoly profit, consumer surplus and pollution costs. 
\begin{align}
\pi _{2}^{t}+CS_{1}^{t}-\psi _{2}^{t}+\delta \left[ \pi
_{1}^{t+1}+CS_{1}^{t+1}-\psi _{1}^{t+1}\right] & =0  \label{eq:14} \\
\pi _{3}^{t-1}+CS_{3}^{t-1}-\psi _{3}^{t-1}+\delta \left[ \pi
_{2}^{t}+CS_{2}^{t}-\psi _{2}^{t}\right] +\delta ^{2}\left[ \pi
_{1}^{t+1}+CS_{1}^{t+1}-\psi _{1}^{t+1}\right] & =0  \label{eq:15'}
\end{align}

\subsubsection{Dynamic inconsistency of the first-best plan}

\label{sec:dynam-incons-path}

As already argued, the presence of a jump variable in the instantaneous
welfare function will likely give rise to a time consistency problem for the
regulator. It would cause the social planner to change their price plan if
they could re-optimise in a future period. To see this, suppose that
currently the regulator follows the plan prescribed by condition \eqref{five}%
. If, however, they were able to deviate from that plan, they would choose
their current price according to \eqref{four}, rather than \eqref{five}.

Mathematically, time inconsistency arises from the presence of the term $%
w_{3}^{t-1}$ in condition \eqref{five}. This term captures the effect of a
change in the current period's price on the previous period's welfare. This
effect will be internalised by a regulator who can precommit to future
prices. However, if a subsequent regulator can re-optimise, he will
disregard periods that, from his viewpoint, have already passed.
Consequently the re-optimised price path will be revised downward, drawing
too much demand away from the preceding period.

Unless the social planner has access to a precommitment device that enables
him to enforce the plan defined by \eqref{four} and \eqref{five}, he will
not be able to attain the first-best outcome because of his incentive to
deviate from it in future periods. When choosing his pricing strategy, a
sophisticated social planner will recognise this problem and account for
future incentives to deviate. The subgame perfect equilibrium of the
intrapersonal game between the current and regulators will deliver a time
consistent sequence of prices.

\subsection{Second-best price path}

\label{sec:second-best-price}

A regulator who is sophisticated (in the sense of \citet{Donoghue1999}) will
take into account the behaviour of their future selves and will choose the
current price accordingly. That is why we now consider a dynamic
intrapersonal game, where the players are the various temporal incarnations
of the social planner. Essentially the current regulator solves a
constrained maximisation problem where future pricing policies are required
to be sub-game perfect. Again, we focus on the Markov-perfect equilibrium of
the regulator's intrapersonal game: the current pricing strategy is assumed
to depend only on the current state of the world: $p^{t}=f(p^{t-1})$.
Furthermore, we assume that the strategy function $f(p)$\ is continuously
differentiable, which eliminates the possibility of an infinite number of
equilibria. However, the above assumptions do not guarantee existence or
uniqueness of a Markov-perfect equilibrium.

Again, we need to specify how the regulator forms their expectations about
future prices. As before, we assume that the social planner has perfectly
rational expectations. That is, he correctly predicts future prices on and
off the equilibrium path. This assumption prevents the existence of a
multiplicity of equilibria \citep{Stokey1981}. Given the focus on Markovian
strategies it implies that $p^{t+1}=f(p^{t})$.

The regulator's MPE price path can be characterised with the help of dynamic
programming. The equilibrium pricing strategy solves the Bellman equation: 
\begin{equation}
W(p^{t-1})=\max_{p^{t}}\Bigl\{w\bigl(p^{t-1},p^{t},f_{e}(p^{t})\bigr)+\delta
W(p^{t})\Bigr\}\qquad \forall :t\geq 1\text{,}  \label{tc:bell}
\end{equation}%
where $W(p)$\ is the social planner's value function. Since MPE strategies
are time invariant, we must also have 
\begin{equation}
f(p^{t-1})=\arg \max_{p^{t}}\Bigl\{w\bigl(p^{t-1},p^{t},f_{e}(p^{t})\bigr)%
+\delta W(p^{t})\Bigr\}.  \label{tc:strat}
\end{equation}%
The recursive formulation of this problem ensures the regulator's pricing
policy will be time consistent.

Suppose that Assumptions \ref{dur:ass1} and \ref{dur:ass2} are satisfied.
Then the method used to derive the monopolist's Euler equation can also
yield the generalised Euler-Lagrange equation of the welfare maximisation
game: 
\begin{equation}
w_{2}^{t}+w_{3}^{t}f_{1}^{t+1}+\delta w_{1}^{t+1}=0.  \label{tc:mpe}
\end{equation}

This equation implicitly defines the second-best price trajectory that would
result from adhering to the time consistent policy function $f(p)$. The term 
$w_{3}^{t}f_{1}^{t+1}$ reflects the intrapersonal strategic effect, i.e. the
effect of current prices on current welfare via future pricing. When the
period-$t+1$\ price is determined, the regulator will not take into account
the negative effect of this pricing decision on period-$t$\ welfare. Thus,
from the current viewpoint, future prices are expected to be suboptimally
low. The period-$t$\ regulator anticipates this behaviour and mitigates
these effects by choosing lower current prices. This implies that the time
consistent prices described by equation \eqref{tc:mpe} will be below the
first-best precommitment prices, thus generating lower welfare.

Using the definition of instantaneous welfare, equation \eqref{tc:mpe} can
be rewritten as 
\begin{multline}
\bigl(\underbrace{\pi _{2}^{t}+f_{1}^{t+1}\pi _{3}^{t}+\delta \pi _{1}^{t+1}}%
_{\mbox{laissez-faire Euler eqn}}\bigr)+\bigl(\underbrace{%
CS_{2}^{t}+f_{1}^{t+1}CS_{3}^{t}}_{\mbox{effect of $p$ on $CS$}}\bigr)
\label{eq:15} \\
-\bigl(\underbrace{\psi _{2}^{t}+f_{1}^{t+1}\psi _{3}^{t}+\delta \psi
_{1}^{t+1}}_{\mbox{effect of $p$ on pollution}}\bigr)=0
\end{multline}%
or, alternatively, 
\begin{equation}
\underbrace{\left[ \pi _{2}^{t}+CS_{2}^{t}-\psi _{2}^{t}\right] }_{%
\mbox{direct effect}}+\underbrace{f_{1}^{t+1}\left[ \pi
_{3}^{t}+CS_{3}^{t}-\psi _{3}^{t}\right] }_{\substack{ \mbox{indirect effect
of $p^t$} \\ \mbox{on
        present via $f^{t+1}$}}}+\underbrace{\delta \left[ \pi
_{1}^{t+1}-\psi _{1}^{t+1}\right] }_{\substack{ 
\mbox{discounted, direct
        effect} \\ \mbox{of $p^t$ on the future}}}=0.  \label{eq:16}
\end{equation}%
Note that, if 
\begin{equation}
CS_{2}^{t}+f_{1}^{t+1}CS_{3}^{t}=\psi _{2}^{t}+f_{1}^{t+1}\psi
_{3}^{t}+\delta \psi _{1}^{t+1},  \label{eq:17}
\end{equation}%
then the time consistent laissez-faire price path could replicate the
regulator's second-best price path. That is to say, if the downward pressure
on the price from the reduction in market power is precisely offset by the
upward pressure from the pollution externality, the laissez-faire outcome
will be efficient. Of course, this coincidence is highly unlikely. It should
also be noted that, if the regulator is unable to precommit, social welfare
could possibly be higher in the absence of government intervention.

We conclude that the social planner's optimal time-consistent price sequence
is not first best. The dynamic consistency problem experienced by the
regulator prevents them from fully correcting the inefficiencies associated
with the pollution emissions of a durable goods monopolist.

%%% Local Variables:
%%% mode: latex
%%% TeX-master: "root"
%%% End:

% LocalWords: externalities precommitment
