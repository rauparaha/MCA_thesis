\chapter{Introduction}

\label{cha:introduction}

Environmental protection has become the political cause c\'{e}l\`{e}bre of
the twenty-first century. Politicians in the European Union and New Zealand
alike are scrambling to seize the moral high ground on the issue of
environmental regulation. For many countries, the idea of a tax on pollution
is very attractive, both economically and politically. In this context it is
opportune to examine any difficulties that may arise in the creation of
optimal regulatory schemes.

Economists have long recognised the problems posed by externalities. There
is a large body of literature on regulation that is designed to overcome the
negative externality imposed on society by polluters. One mechanism features
prominently in the environmental literature is taxation. Pigouvian taxation
is not only theoretically effective for mitigating externalities; it is also
a mechanism that is relatively easy for regulators to implement. Most
governments already have various taxes in place, so creating a pollution tax
is a task for which the administrative infrastructure already exists.

Unfortunately, governments can face certain difficulties in implementing
efficient taxation. This thesis will examine regulating an industry where
agents exhibit dynamically inconsistent behaviour. In such an industry, a
regulator who maximises some welfare function that accounts for industry
profits will find that his policies may also suffer from time consistency
problems. If the government cannot precommit to future taxes, it will be
unable to achieve the first-best outcome for society. We propose a taxation
mechanism which allows the regulator to overcome his dynamic inconsistency
problem, and thus realise the first-best regulatory outcome.

Chapter \ref{cha:theor-fram} canvasses the theoretical framework of dynamic
inconsistency that underlies the proposed mechanism. Chapter \ref%
{cha:problem} constructs a model of a polluting monopolist producing a
durable good and a regulator attempting to address the pollution problem.
This model is used to demonstrate the that the inability of the regulator to
precommit to future actions will prevent him from achieving the first-best
outcome. Chapter \ref{cha:proposition} describes the proposed Pigouvian
taxation mechanism. The method by which it overcomes the dynamic
inconsistency problem and implements efficiency is explained. Since the
model does not permit a closed-form solution, we develop a numerical example
in chapter \ref{cha:computation} to investigate the effect of parameter
variations.

%%% Local Variables:
%%% mode: latex
%%% TeX-master: "root"
%%% End: