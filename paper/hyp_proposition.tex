
\chapter{A regulatory game with hyperbolic discounting}
\label{cha:proposition-1}


Intuitively it should be possible to solve a quasi-hyperbolic
discounting problem through `delegation' of the pricing
decision.

Preferences which include quasi-hyperbolic discounting give rise to
time inconsistent behaviour. However, the inconsistency is between
only the current and next periods. The agent's intended behaviour will
remain consistent across future periods. Thus, by eliminating the
direct effect of today's decision on future payoffs, it should be
possible to render the decision maker consistent.

\section{Regulation with delegation}
\label{sec:regul-with-deleg}

The delegation game involves the regulator setting a tax rate for
pollution simultaneously with the monopolist's choice of output in
each period. Both the tax and the choice of output are feedback
strategies. We consider only a linear tax, however there are many
possible ways to levy it. It could also be levied on either emissions
or upon the stock of pollution. In this section we consider only the
case of a tax on emissions for consistency with Part \ref{part:jump}.

The case of a tax on emissions will be considered first and then the
problem with a pollution tax will be solved.

\subsection{The welfare function}
\label{sec:welfare-function-1}

The regulator's problem changes for two reasons: first, because he
gains revenue from taxation and, secondly, because we introduce a cost
to changing the tax rate. Economists are often criticised by
policymakers for excluding the costs of implementation when they
recommend taxes. Here, we explicitly include the costs of implementing
and modifying tax schemes in the regulator's welfare function.

Suppose that the tax is levied on emissions and the revenue from the
tax is given to consumers as a lump sum. Since the marginal utility of
income to consumers is $1$, the value of the revenue in the welfare
function is equal to the cost levied on the monopolist. Hence, the tax
is a simple transfer of surplus and does not change total welfare.

The value of the tax revenue is $\tau_t x_t$, where $\tau_t$ is the
period $t$ tax rate. Let the cost of changing policies be $\kappa
\rho(\tau_t,\tau_{t-1})$.\footnote{A plausible, specific functional
  form might be $\left(\tau_t - \tau_{t-1}\right)^2$, as in Part
  \ref{part:jump}.} Then the welfare function becomes
\begin{align}
  \label{eq:56}  w(x_t,k_t,\tau_t,\tau_{t-1}) = u(x_t) - C(x_t) - \phi(k_t) - \kappa \rho(\tau_t,\tau_{t-1}).
\end{align}
The welfare function is not affected by the level of revenue generated
by the tax, but it is altered by changes in the tax rate. This
introduces a `stickiness' to the tax rate. Note that, if $\kappa=0$,
then there are no costs of policy adjustment and the welfare function
does not depend upon the tax rate.

\subsection{The profit function}
\label{sec:profit-function}

With taxation the monopolist's instantaneous profit becomes
\begin{equation}
  \label{eq:57} \pi_t =  u'(x_t).x_t - C(x_t) - \tau_tx_t.
\end{equation}

\subsection{The taxation game}
\label{sec:delegation-game}

We now construct the regulation game for this problem that mirrors the
game discussed in Chapter \ref{cha:proposition}.

\subsubsection{State variables and strategies}
\label{sec:state-variables-1}

The state variables for the game are the current tax rate,
$\tau_{t-1}$, and the stock of pollution, $k_t$. Note that the tax
rate is the previous period's: the current period's tax rate cannot be
a state variable because it is chosen in the current period.

The monopolist chooses output and the regulator sets the tax rate
concurrently. Let the MPE strategy of the monopolist be $x_t =
h(\tau_{t-1},k_t)$ and the MPE strategy of the regulator be $\tau_t =
g(\tau_{t-1},k_t)$.

\subsubsection{The regulator's problem}
\label{sec:regulators-problem}

Since the regulator discounts in a quasi-hyperbolic fashion, his
preferences are non-stationary. The current period's value function is
\begin{multline}
  \label{eq:58} U(\tau_{t-1},k_t) = \max_{\tau_t} \bigg\{ w\big(
    h(\tau_{t-1},k_t),k_t,\tau_t,\tau_{t-1}\big) \\+ \beta\delta V
  \big( \tau_t,\theta k_t + h(\tau_{t-1},k_t) \big) \bigg\}.
\end{multline}
The value function for future periods is
\begin{multline}
  \label{eq:59}  V(\tau_{t-1},k_t) = w\big(
  h(\tau_{t-1},k_t),k_t,g(\tau_{t-1},k_t),\tau_{t-1}\big) \\ +
  \delta V \big( g(\tau_{t-1},k_t),\theta k_t + h(\tau_{t-1},k_t)
  \big).
\end{multline}

\subsubsection{The monopolist's problem}
\label{sec:monopolists-problem}

Since the monopolist discounts exponentially, he has a stationary
value function:
\begin{equation}
\label{eq:73}
  \Pi(\tau_{t-1},k_t) = \max_{x_t} \left\{
    \pi(x_t,g\left(\tau_{t-1},k_t)\right) + \delta \Pi \left(
      g(\tau_{t-1},k_t), \theta k_t + x_t \right) \right\}.
\end{equation}

\subsubsection{Equilibrium strategies}
\label{sec:equil-strat-1}

Solving the system of equations specified by equations
\eqref{eq:58}--~\eqref{eq:73} gives the generalised Euler-Lagrange
equations characterising the optimal strategies for each player.

The Euler-Lagrange equation characterising the regulator's strategy is
\begin{multline}
  \label{eq:30}
  \frac{w^{t-1}_3g^{t-1}_1 - \beta\left( w^{t-1}_1h^{t-1}_1 +
      w^{t-1}_3g^{t-1}_1 + w^{t-1}_4 \right)}{\delta h^{t-1}_1}  -
  \frac{w^{t-2}_3}{\delta^2 h^{t-1}_1} \\=
  \beta ( w^t_1h^t_2 + w^t_2 + w^t_3g^t_2 ) - g^t_2w^t_3 \\
  +\frac{\left( \theta + h^t_2 \right)}{h^t_1} \left[ w^t_3g^t_1 -
    \beta\left( w^t_1h^t_1 + w^t_3g^t_1 + w^t_4 \right) -
    \frac{w^{t-1}_3}{\delta} \right].
\end{multline}
See Appendix \ref{sec:regulators-problem-1} for the derivation.;

In order to compare to the first-best solution let us simplify this
expression to the special case where $\kappa=0$ and there is no cost
of policy change. This can be done by substituting in the partial
derivatives of the welfare function:
\begin{align}
  \label{eq:67}
  w^t_1 &= u'(x_t) - c'(x_t) \\ \label{eq:68} w^t_2 &= -\phi'(x_t)
  \\ \label{eq:69} w^t_3 &= 0 \\ \label{eq:70} w^t_4 &= 0.
\end{align}
After substitution the Euler equation, \eqref{eq:30} reduces to
\begin{equation}
  \label{eq:71}
  \frac{u'(x_{t-1}) - c'(x_{t-1})}{\delta} - \phi'(x_t) - \theta \big(
  u'(x_t) - c'(x_t) \big) = 0.
\end{equation}
Multiplying by $\delta$ and shifting the equation forward one period
gives
\begin{equation}
  \label{eq:72}
  u'(x_t) - c'(x_t) - \delta \phi'(x_{t+1}) - \delta\theta \big(
  u'(x_{t+1}) - c'(x_{t+1}) \big) = 0
\end{equation}
which replicates the result of equation \eqref{eq:45}, the
precommitment outcome. Thus, the regulation game replicates the output
path of the first-best.

For the monopolist's choice of $x_t$ the relevant Euler-Lagrange
equation is
\begin{equation}
\label{eq:84}
  \pi^t_2g^t_1 + \frac{g^t_1}{g^t_2} \left( \theta \pi^t_1 -
    \pi^t_2g^t_2 - \frac{1}\delta\pi^{t-1}_1 \right) -
  \frac{\theta\pi^{t-1}_1 - \pi^{t-1}_2g^{t-1}_2 -
    \frac{1}\delta\pi^{t-2}_1}{\delta g^{t-1}_2} = 0.
\end{equation}
See Appendix \ref{sec:monopolists-problem-1} for a derivation.



% The derivatives of the profit function can now be found and
% substituted in:
% \begin{multline}
% \label{eq:80}
% (-x_t)g^t_1 + \frac{g^t_1}{g^t_2} \bigg( \theta \bigg( u''(x_t)x_t +
% u'(x_t) - C'(x_t) - \tau_t \bigg) \\ - (-x_t)g^t_2 -
% \frac{1}\delta\bigg( u''(x_{t-1})x_{t-1} + u'(x_{t-1}) - C'(x_{t-1}) -
% \tau_{t-1} \bigg) \bigg) \\ - \frac{\theta\bigg( u''(x_{t-1})x_{t-1} +
%   u'(x_{t-1}) - C'(x_{t-1}) - \tau_{t-1} \bigg) - (-x_{t-1})g^{t-1}_2
%   \\- \frac{1}\delta\bigg( u''(x_{t-2})x_{t-2} + u'(x_{t-2}) -
%   C'(x_{t-2}) - \tau_{t-2} \bigg)}{\delta g^{t-1}_2} = 0
% \end{multline}


%%% Local Variables:
%%% mode: latex
%%% TeX-master: "root"
%%% End:
