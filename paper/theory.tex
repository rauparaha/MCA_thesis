\chapter{Theoretical framework}

\label{cha:theor-fram} 
% Note: \textit{The aim of this section is to introduce the concepts
%   that underlie the problem. By the end of the section the problem
%   should be clear and of interest to the reader. The key concept to
%   explain is dynamic consistency. The progression of the chapter
%   leading up to this must discuss
%   \begin{itemize}
%   \item Dynamic games
%   \item Feedback vs open loop equilibria
%   \item Markovian strategies and the Folk theorem
%   \item Dynamic consistency vs time-consistency
%   \item jump variables and the sufficiency condition for time-consistency.
%   \end{itemize}
%   It is also important to stress the difficulties of regulating a
%   dynamically inconsistent polluter. The consequences of failing to
%   remedy dynamic inconsistency are very high when regulation of
%   pollution is involved. Emissions regulation must somehow be painted
%   as an integral part of the motivation for this study rather than as
%   a vehicle for examining dynamic consistency.}

The following section discusses the theoretical framework of the thesis. The
game theoretic context of the problem is explained, as is the reason for
choosing to focus on the regulation of pollution.

Section \ref{sec:dynamic-games} briefly covers the theory of dynamic games
and specifies the type of games that will be considered. Section \ref%
{sec:dynamic-consistency} explores the notion of dynamic inconsistency. Here
the type of dynamic inconsistency that affects the model in chapter \ref%
{cha:problem} will be delineated. Section \ref{sec:caus-regul-dynam} surveys
the types of situations such inconsistency problems may occur.

The innovation of this thesis is to design a taxation mechanism that will
aid a government which regulates a polluter while experiencing dynamic
inconsistency. The aim is to show that this regulator can use taxes as an
instrument that provides polluters with efficient incentives even when his
welfare maximisation problem contains a jump variable.

\section{Dynamic games}

\label{sec:dynamic-games}

\subsection{Features of dynamic games}

\label{sec:feat-dynam-games} Static game theory studies the interaction of
strategic agents in a so-called `one-shot' framework. Each player makes a
one-time decision, and there is no future play. Such games can be described
as static, since they do not have any dynamic element. In reality, many
important decisions occur in the context of ongoing relationships. As a
result, the idea of a repeated game (also called a supergame) was developed.

In a repeated game, the one-shot game is played multiple times by the same
players. These games may produce results unobtainable in static games (e.g.
cooperation in the infinite horizon Prisoners' Dilemma). However, they
cannot capture some important features of many observed relationships,
because the context of the play is the same in every period. In real life,
it is often the case that one's actions today affect one's possible future
actions, and thus potential future payoffs. Repeated games do not acocunt
for such inter-temporal linkages. It is these interactions that dynamic
games seek to model. In this thesis, the term `dynamic game' will refer to
games with state dynamics. Repeated games are not within the ambit of
dynamic games as the term is used here.

A dynamic game is modelled as a dynamical system in which the state of the
world changes over time in response to players' actions. The state variables
describe the current state of the system. They may influence the payoffs, or
the action space, of the players. The state changes over time according to a
pre-defined law of motion, which may depend upon the players' actions. In
such a system, players' current actions will affect their future payoffs
through the state. In addition to the standard static strategic effects,
these settings allow for the possibility of inter-temporal strategic
effects. Sometimes they also give rise to intra-personal, inter-temporal
strategic effects: an agent may have a conflict with their future `self'.

In general, the most popular solution concept in games is that of Nash
equilibrium. In dynamic games, various refinements are used to rule out Nash
equilibria that may be considered implausible. A couple of these will be
explained later in this chapter.

\subsection{Open loop vs feedback strategies}

\label{sec:open loop-vs} There are two common ways to model players'
behaviour in a dynamic game: players could either precommit up front to
their future course of action, or they could choose their action in each
period based on the current state. The former strategies are known as open
loop strategies, as they are non-responsive to changes in the state. The
latter strategies are called feedback strategies, because a change in the
state can affect the player's actions.

An equilibrium in open loop strategies is justifiable only if players have
precommitment power. Indeed, in an open loop equilibrium, players will not
change their action in response to deviations from the equilibrium by other
players. Essentially, open loop games are static games with a
multi-dimensional action space: instead of choosing one action, at the
beginning of time the players choose actions for all periods.

In open loop equilibria, players choose their future actions, while taking
the future actions of their opponents as given. Constructing strategies in
this fashion means that the prescribed actions will often be sub-optimal if
the state variable deviates from the anticipated equilibrium path. Thus,
open loop strategies are generally not subgame perfect: they will not be
optimal in all possible subgames.

When modelling a situation in which players have the ability to respond to
deviations of the state, or in other players' strategies, it is more
appropriate to consider feedback strategies. This approach specifies the
player's strategy as a decision rule, i.e. a function of the state. It is
computed while taking as given the decision rules of the other players.
Because such strategies are optimal for all possible states, by construction
they will be subgame perfect.

\subsection{State variables}

\label{sec:state-variables} Unlike other types of games, in dynamic games
players' payoffs can be affected by state variables which change over time.
The choice of state variables is an important modelling decision that can
have a marked effect upon the outcome of the game.

There are two important types of state variables: first, those that have a
direct physical or technological impact upon the game and, secondly, those
that affects the psychology of agents and, through it, their behaviour. The
first type includes components of the game structure that affect players'
payoffs directly, such as pollution stock levels. The second type includes
variables that are not payoff-relevant, such as the history of play.

In repeated games, the Folk Theorem establishes that there exist a
multiplicity of equilibria, provided that players have long memories %
\citep{Fudenberg1986}. Current research demonstrates that the same result
holds for dynamic games in which strategies are allowed to depend on state
variables that are not payoff-relevant \citep{Haurie1987,Ausubel1989,Gul1986}
. The reason is that equilibrium behaviour is influenced by agents' beliefs
about the consequences of deviations from the equilibrium path. Including
state variables that affect players' psyche, such as the history of play,
can give rise to many possible belief structures. This will result in
multiplicity of equilibria. Because it is difficult to draw economic
conclusions without a unique prediction, it is common to restrict the state
variables to those which directly affect players' payoffs. This restriction
does not guarantee a unique equilibrium, but it does reduce the number of
possible equilibria.

\subsection{Markov strategies}

\label{sec:markov-strategies} Strategies that are based on information sets
which include only payoff-relevant information are known as Markov
strategies. The assumption of Markov strategies implies that players are
unable to observe events that happened in the deep past. The history of play
is thus summarised through its effect upon the state variables.

When players' past actions are not directly observable, it is difficult to
punish past deviations from the equilibrium. Effectively, the restriction to
Markov strategies rules out the use of trigger strategies. The range of
potential equilibria is therefore significantly narrowed.

While Markov strategies offer a way to limit the number of possible
equilibria, they may not always be appropriate. For the Markovian
restriction to be reasonable, players must be unlikely to use trigger
strategies. In the regulation games studied here, we claim that players are
unlikely to use trigger strategies. It seems implausible that a government
would choose to punish a firm for deviations from the desired output level
by penalising it for many periods in the future. When governments choose
their tax rules, they enshrine them in legislation. The tax authority then
applies the tax rule as specified in the legislation. Since the creation of
the rule and its application are usually separate activities, taxation rules
are unlikely to be reactionary. Governments also prefer to appear
even-handed in their policies. A trigger strategy relies on the threat of
harsh punishments to enforce an efficient equilibrium. Given that a
deviation from the equilibrium output level is not illegal, it is unlikely
that the government would wish to punish a firm for such a deviation. It is
more reasonable to believe that the level of taxation will depend on the
current state of the industry, but not upon the firms' history of actions.
Thus, the use of Markov strategies in this thesis seems justified.

\section{Dynamic consistency}

\label{sec:dynamic-consistency}

\subsection{Dynamic consistency and precommitment}

\label{sec:dynam-cons-prec} The extension of game theory to dynamic games
opened up a range of interesting new problems. Among those problems is the
issue of dynamic consistency, which was first explored in the context of
economic policy by \citet{Kydland1977}. A plan is said to be dynamically
consistent if a player has no incentive to deviate from it at any time in
the future. Conversely, a dynamically inconsistent plan is one which, while
optimal when conceived, the player will choose to deviate from in the future.

A dynamically inconsistent agent will be unable to adhere to an optimal plan
without the aid of some precommitment device. Some of these devices are
usually enforced by a third party. On a personal level, people rely on
family members and friends to help them follow a chosen course of action; on
a business level, the legal system enforces contractual commitments.
However, precommitment power is harder to come by when there is no external
`enforcer' available. Governments, in particular, may find it very difficult
to bind themselves, since any legislation that they pass can be overturned
by future legislators. This thesis will examine the problem of a dynamically
inconsistent regulator in some detail and propose a method by which a
regulator can overcome their dynamic inconsistency.

\subsection{Time-consistency and perfection}

\label{sec:time-cons-perf} If precommitment is not possible, then an
equilibrium involving dynamically inconsistent strategies is implausible. No
agent will be willing to rely upon promises that they expect to be broken.
Hence, strategies must be dynamically consistent to generate plausible
equilibria. We now examine in greater detail precisely what is meant by
dynamic consistency. There are two ideas of dynamic consistency that are
relevant to this thesis: `time-consistency' and `subgame perfection'.

Time-consistency is the weaker of the two requirements. Any equilibrium that
is not dynamically inconsistent is time consistent. Given that no agent has
reneged in the past, and none are expected to renege in the future, no agent
has an incentive to unilaterally renege on a time-consistent equilibrium.

A subgame perfect equilibrium must satisfy a more stringent test. subgame
perfection requires that a strategy be optimal regardless of past deviations
in either the state or in the other players' actions. This must be true for
all possible values of the state variable and across all time periods. As
such, a perfect strategy will usually depend on the state variable.

Open loop strategies are announced by agents at the beginning of the game,
and so describe actions as functions of time and the initial state. Thus, a
deviation from the expected path by any player will not change the actions
dictated by an open loop strategy. This is likely to cause the open loop
strategy to be sub-optimal in the periods following the deviation.
Therefore, the player will have an incentive to deviate from it. It follows
that the open loop strategy is not subgame perfect. Note that the strategy
may still be time-consistent even if it is not dynamically inconsistent, as
long as play remains on the equilibrium path.

In dynamic games, subgame perfection may be attained with state-dependent,
feedback strategies. Since perfection implies time-consistency (even though
the converse is not true), a perfect feedback strategy will also be time
consistent. Hence, time-consistent strategies can be either open loop or
feedback.

\subsection{Sufficient conditions for time-consistency}

\label{sec:suff-cond-time} Feedback strategies are designed to be optimal in
all states. Thus, an equilibrium in feedback strategies is subgame perfect
by construction. Open loop equilibria are rarely subgame perfect, but are
often time-consistent along the equilibrium path. Bellman's Principle of
Optimality states that the continuation of an optimal strategy is optimal in
all states that arise from past optimal behaviour \citep{Bellman1957}. Thus,
if the initial state is exogenously given, and players all choose optimal
strategies, the resulting equilibrium will be time-consistent %
\citep{Karp1993a}.

The difficulty with this result is that it depends upon an exogenously
specified initial state. If the initial state is not exogenous, then the
current state must be a function of future actions. A variable whose value
depends upon future events is known as a `jump variable'. Problems with jump
variables are not covered by the Principle of Optimality and rarely have
time-consistent open loop solutions. However, the absence of a jump variable
is sufficient to ensure time-consistency of an open loop equilibrium.

The intuition behind the above statement is fairly straightforward. The
presence of a jump variable suggests that the current state and the current
payoff depend upon expectations about the future. Thus, in period $t$,
agents need to form expectations about actions and payoffs in period $t+n$,
so that they can compute their optimal action. Time inconsistency can arise
in two ways: first, if these expectations are not fulfilled then the period $%
t$ action becomes sub-optimal, and so the period $t+n$ action will be
different from what was anticipated. Even if expectations are fulfilled, the
period $t+n$ action is still likely to differ from the expected action. This
is because the $t+n$ action has an effect upon the period $t$ payoff.
Re-optimising in period $t+n$ would disregard the `inter-temporal
externality' on the previous periods' payoffs, since they are now sunk.
Thus, in period $t+n$, the optimal action is unlikely to be the same as was
expected in period $t$. As a result, the open loop equilibrium calculated in
period $t$ will be time inconsistent.

Dynamic inconsistency is common in settings where governments regulate
non-strategic but forward looking agents \citep{Chari1988}. When making
decisions about the current period, such agents take future payoffs into
consideration. If current payoffs are affected by future actions, then the
problem of a subsequent regulator will differ from that of the current one.
This implies that the regulator's optimal plan will change over time, and so
the regulator will be dynamically inconsistent.

\subsection{Implications of dynamic inconsistency for regulation}

\label{sec:cons-dynam-incons} The problem considered in this thesis will
exhibit dynamic inconsistency due to the presence of a jump variable in the
regulator's objective function. As a result of this, the regulator is unable
to obtain the first-best price path. In the subsequent chapters, we will
focus on this inefficiency. But before describing the specifics of the
model, it is worth canvassing the likely consequences for a dynamically
inconsistent regulator. Note that the assumptions in the following
paragraphs are made purely for exposition purposes. All assumptions
underlying the formal model will be explicitly stated in chapter \ref%
{cha:problem}.

Imagine that the regulator's objective function includes a jump variable.
That is, welfare depends upon the regulator's future actions. In each
period, the regulator chooses the rate of a tax that is imposed on a
polluter. If the tax rate is expected to fall in the next period, the
polluter will inter-temporally substitute away from current production
towards future production; hence, production in the current period will
decrease. Suppose that the fall in the current period's profits and consumer
surplus outweighs the decrease in pollution. Then the current period's
welfare will decrease. To prevent this, the regulator must commit to setting
high taxes in future periods. The expectation of high future taxes would
remove the incentive of the polluter for inter-temporal substitution.
However, once these future periods arrive, the regulator will no longer be
concerned with the effect of his tax choice upon past welfare. Thus, the
regulator would revise the tax downward if they have the opportunity to do
that.

A sophisticated regulator will anticipate future temptations to decrease
taxes. If they are unable to commit to future policies, they will set the
current tax strategically to counter future incentives for tax reductions.
They achieve that by setting a low current tax rate. The essence of the
problem is that the polluter is under-producing in the current period,
relative to the future period. This problem can be corrected by decreasing
current taxation to remove the incentive to shift production to the future.
As a result, a sophisticated regulator who is unable to precommit will
under-tax the polluter relative to the social optimum.

Under-taxation of a polluter implies over-pollution. Since pollution and its
consequences are some of the most pressing problems facing modern
industrialised societies, designing an effective mechanism to control it is
a problem of great importance. Correcting a potential policy flaw that could
lead to significant over-pollution is far more than an academic exercise.

\section{Causes of regulatory dynamic inconsistency}

\label{sec:caus-regul-dynam}

\subsection{Inconsistency of the welfare function}

\label{sec:incons-welf-funct}

A regulator's welfare function consists of four components: firms' profits,
consumer surplus, externalities and government revenue. For now, revenues
will be ignored. We abstract from them in order to focus on corrective,
rather than revenue gathering, regulation.

In a regulated industry that exhibits dynamic inconsistency, firms' profits
will often contain a jump variable. It is also possible that consumer
surplus may contain a jump variable, provided that the inconsistency in the
industry is caused by demand-side behaviour. Jump variables in either of
these functions can cause the regulator's welfare function to exhibit
inconsistency.

Note that inconsistency in the profit function will not always be
transferred to welfare: if the jump variable is in the inverse demand
function alone, and firms' revenues are simply a transfer from consumers to
firms, then the jump variable will not appear in welfare. However, this will
not be true if the regulator weighs profits and consumer surplus
differently. In this case, expectations about future variables will still be
present in the welfare function and the regulator will be inconsistent.

\subsection{Industries in which dynamic inconsistency arises}

\label{sec:industr-which-dynam}

There are numerous examples of industries where dynamic inconsistency arises
from a jump variable in a firm's profit function. Most commonly, it is found
where durable or addictive goods are being produced, or where an exhaustible
resource is being extracted.

\subsubsection{Exhaustible resources}

\label{sec:exha-reso} Economists have been interested in exhaustible
resources since Harold Hotelling's seminal paper \citep{Hotelling1931}. In
this paper, Hotelling shows that the pricing of an exhaustible resource
depends upon expectations of subsequent prices. This dependence upon future
decisions introduces a jump variable the profits of firms who utilise the
resource. In particular, \citet{Karp1993a} show that a firm with monopsony
power who purchases an exhaustible resource will be dynamically inconsistent
if they face either a competitive fringe of consumers, or increasing
extraction costs.

\subsubsection{Addictive goods}

\label{sec:addictive-goods}

The theory of rational addiction suggests that an addictive good can be
modelled as a commodity whose current consumption increases the marginal
benefit of future consumption \citep{Becker1988}. If expected future prices
are low, then consumption is affected in two ways. First, expected future
consumption will rise as a direct result of the lower expected price.
Secondly, current consumption will rise as consumers attempt to increase the
benefits that they will reap from the low future prices. Thus, a firm
selling an addictive good can affect current demand by manipulating
expectations about its future pricing strategy. This introduces a jump
variable into the firm's profit function and causes the firm to act in a
dynamically inconsistent fashion.

\subsubsection{Durable goods}

\label{sec:durable-goods} A durable good it one which is not consumed
instantly, but continues to provide value for an extended period of time.
When an individual decides to purchase a durable good they must weigh the
benefit of purchasing in the current period against possible price
reductions if they delay purchase until a later period. A firm selling
durable goods, like a firm selling addictive goods, can influence the
current period's profit by manipulating consumers' expectations about future
prices. A high expected future price will induce more consumers to buy in
the current period. The future price is a jump variable in the firm's profit
function and induces dynamic inconsistency.

%%% Local Variables:
%%% mode: latex
%%% TeX-master: "root"
%%% End:

% LocalWords:  precommitment equilibria precommit externalities