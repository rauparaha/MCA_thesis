\chapter{Proposition}

\label{cha:proposition}

The previous section demonstrated that if there is a jump variable in firm
profits and the regulator cannot precommit to future prices, he is unable to
follow the first-best price path. That is because the polluter's time
consistency problem is transferred to the regulator. In order to maintain
time consistency of his policies, the regulator can only implement a second
best outcome. The first-best level of welfare is feasible only if the
government could credibly precommit to future actions.

The problem of gaining commitment power, where it is not obviously
available, has been explored in the literature on strategic delegation. In
this chapter we study how the idea of strategic delegation has been used in
the context of duopoly games to avail agents of precommitment power. We then
combine the concept of strategic delegation with Pigouvian taxation to
create a taxation mechanism that overcomes the regulator's time consistency
problem.

\section{Strategic delegation}

\label{sec:strategic-delegation}

The strategic delegation literature claims that the separation of ownership
and management can be used as a means of gaining a strategic advantage in
imperfectly competitive markets. Oligopolists who compete in a Cournot
setting would each like to be in the position of a Stackelberg leader. By
delegating output decisions, they can use managerial contracts to gain a
first-mover advantage. If the owners can set the managerial wage contracts
before output decisions are made, they can provide their managers with
incentives for aggressive production. Thus, wage contracts can have
commitment power: delegation will provide firms with a first-mover advantage.

The idea of strategic delegation began with the papers of \citet{Sklivas1987}
and \citet{Fershtman1987}, in which they suggest that each duopolist could
increase their profits by delegating output decisions to a manager. The
manager's behaviour is incentivised through a remuneration contract. The
papers show that the decision to delegate managerial control is individually
rational and a dominant strategy for each firm. This line of work has been
continued by many authors, including \citet{Miller2001}, \citet{Basu1993}
and \citet{Baye1996}.

In a different strand of research, \citet{Rogoff1985} shows that a
government too can benefit from the precommitment power of delegation. He
studies a macroeconomic model of monetary policy. Agents' rational
expectations imply the presence of jump states in welfare, giving rise to a
time consistency problem for the government. Thus, in the absence of
precommitment, the government's monetary policy decisions tend to exhibit
inflationary bias. Rogoff argues that the appointment of a central banker
with a particular set of preferences would allow the government to credibly
commit to socially optimal inflation \ However, such one-shot delegation
would require the regulator to find a third party whose preferences are
socially optimal. The difficulties this entails are obvious, so such a
solution cannot be considered a practical.

Similarly, we argue that `delegation' could provide the government with the
necessary precommitment power to achieve the first-best outcome when
regulating a durable goods monopolist. We combine the approaches of %
\citet{Rogoff1985} and \citet{Sklivas1987}. In our model, pricing decisions
are 'delegated' to the monopolist, while the regulator uses pecuniary
incentives to influence future prices. The monopolist is induced to follow
the regulator's preferred price path with the help of a taxation mechanism.
The purpose of the tax is to provide the producer with socially optimal
incentives. Our contribution is to show that taxes not only redress the
inefficiencies arising from pollution and market power, but also serve as a
commitment device by decoupling the regulator's decision from the pricing
decision.

\section{Optimal taxation}

\label{sec:optimal-taxation}

Using a delegation tax game to overcome dynamic inconsistency has three key
advantages over Rogoff's one-shot delegation approach:

\begin{description}
\item[Ease of implementation] Taxation is a type of regulation that is
already performed by the government and so the institutions are already in
place to implement this policy. The regulator only needs to adjust the tax
rule to ensure time-consistent implementation.

\item[Dynamic robustness] A regulator who has a time consistency problem
will have an incentive to intervene in the future if they are not insulated
from the pricing decision. We show that our proposal attains the first-best
outcome even when the government is able to alter the taxation policy in
future periods.

\item[Insensitivity to managerial preferences] In contrast to the approach
taken by Rogoff, our model does not require delegation to a party with a
particular set of preferences. Pricing decisions are made by the monopolist
themselves and the taxation mechanism will provide them with correct
incentives regardless of their demand and cost structure.
\end{description}

\subsection{A regulatory model}

\label{sec:regulatory-model}

We now extend the model described in the previous chapter to include a
taxation mechanism and show how it overcomes the regulator's time
inconsistency problem.

To maximise social welfare, the social planner must eliminate the
inefficiencies generated by the pollution externality, the market power and
the dynamic inconsistency of the monopolist. The government can motivate him
to follow the socially optimal price path by using Pigouvian taxes that
increase firm costs. Rather than directly choosing prices, the social
planner now maximises welfare by choosing a tax policy. The timing of the
taxation game discussed below implies that the regulator cannot affect the
current choice. This enables him to resolve his time inconsistency problem.

\subsubsection{Pigouvian taxes}

\label{sec:tax-rule}

In this model we consider a flow pollution externality $\psi (x^{t})$. The
Pigouvian tax could be levied on either output or pollution. We assume that
the government taxes emissions, but our results would be similar if output
was taxed instead. Thus, the monopolist's period-$t$\ tax obligation can be
written as 
\begin{equation}
\Omega ^{t}=\tau ^{t}\psi (x^{t}).
\end{equation}%
Note that $\tau ^{t}$ might vary over time. In our model, this would happen
if the value of the state variable changes throughout the game.

The monopolist's instantaneous profit, net of taxes, is given by 
\begin{equation}
\pi ^{t}(p^{t-1},p^{t},p^{t+1},\tau ^{t})=p^{t}x^{t}-C(x^{t})-\tau ^{t}\psi
(x^{t}).
\end{equation}

\subsubsection{Tax revenues}

\label{sec:reven-from-taxat}

The tax revenue raised by the government is defined as the sum paid by the
monopolist. However, the regulator's valuation of this revenue may not be
the same as its monetary value. To recognise that the government and the
monopolist may value tax revenues differently, we assume that the welfare
benefit from the tax revenues is $\alpha \tau ^{t}\psi (x^{t}),\quad \alpha
\in \lbrack 0,1]$. If $\alpha =0$\ tax revenues have no social benefit,
while $\alpha =1$ indicates that the government and the monopolist value the
revenue equally.

We do not hypothesise what might be done with the tax revenues since they
may not remain within the industry. However, in order to remain
revenue-neutral, they could be returned to the monopolist as a lump-sum
transfer. Provided that the transfer is not dependent upon the monopolist's
actions, they would not have an impact on the price path chosen or the
efficiency of the market.

\subsubsection{Costs to policy adjustment}

\label{sec:costs-policy-adjustm}

In many real-world situations, government policy changes are not costless.
There could be costs to changing the tax regime in the consultation, policy
development and political manoeuvring that must be done. Analytically this
implies that previous tax policies may affect current welfare. Consequently, 
$\tau ^{t-1}$ would be a payoff-relevant state variable in period $t$. 

We assume that the larger the deviation from the status quo, the larger the
welfare cost. To model this consideration, we include the term 
\begin{equation}
\theta \big(\tau ^{t}-\tau ^{t-1}\big)^{2}  \label{eq:18}
\end{equation}%
in period-$t$ instantaneous welfare. That is, the cost of changing the tax
rate increases proportionately to the square of the change.\footnote{%
Any convex function would suffice but the use of a quadratic function
simplifies the derivations with little loss of generality since the
functional form is not central to the proposed mechanism.} This
specification ensures that the adjustment cost is positive. Moreover, it
reflects the difficulty of enacting significant changes in regulations. The
coefficient $\theta $ allows us to capture the importance of these costs
relative to consumer surplus and profits.

The regulator's instantaneous payoff is thus 
\begin{multline}
w^{t}\big(p^{t-1},p^{t},p^{t+1},\tau ^{t-1},\tau ^{t}\big)=\underbrace{\pi
^{t}\big(p^{t-1},p^{t},p^{t+1}\big)}_{\mbox{Gross
      profits}}+\underbrace{CS^{t}\big(p^{t-1},p^{t},p^{t+1}\big)}
_{\substack{ \mbox{Consumer} \\ \mbox{surplus}}}  \label{eq:19} \\
-\underbrace{\psi ^{t}\big(p^{t-1},p^{t},p^{t+1}\big)}_{\substack{ %
\mbox{Pollution} \\ \mbox{externality}}}-\underbrace{(1-\alpha )\tau
^{t}\psi ^{t}\big(p^{t-1},p^{t},p^{t+1}\big)}_{\substack{ \mbox{Net cost of}
\\ \mbox{taxation}}}-\underbrace{\theta \big(\tau ^{t}-\tau ^{t-1}\big)^{2}}
_{\substack{ \mbox{Cost of
        policy} \\ \mbox{adjustment}}}
\end{multline}%
As already discussed, welfare maximization with respect to prices is likely
to create time inconsistency issues since period-$t$ welfare would depend on
the jump variable $p^{t+1}$. However, in the tax games analyzed below, the
regulator now chooses taxes, while leaving pricing decisions to the durable
goods monopolist.

\subsection{Timing}

Having established the implications of taxation for welfare and profits, we
now define the timing of the interactions between the two parties. We
consider a setting where the monopolist and the regulator make their
decisions simultaneously in each period. The assumption of simultaneity
describes a situation in which the monopolist is unaware of the regulator's
current tax rate prior to setting the price. This describes the common
situation in which firms make decisions before the details of current tax
policies have been announced to the public. It is likely that there would be
some scope for discussion of risk and option value in this model. We leave
these issues for future research.

The timing assumption is crucially important for our results. The
simultaneity of decisions causes the monopolist's current price choice to be
independent of current taxes. This removes the jump state from the
regulator's objective and remedies his time inconsistency problem. If the
monopolist knew the current period's taxes, the regulator could influence
the current period's price. The regulator's current payoff would then depend
on his next period's choice and thus his time inconsistency problem will
remain.

\subsection{The equilibrium of the regulation game}

\label{sec:regulation-game}

In this section we formulate the above problem as a dynamic game and then
solve for the subgame-perfect equilibrium tax and pricing strategies. We
model regulation as a game between the various temporal incarnations of the
regulator and the monopolist. When the regulator sets the current tax rate $%
\tau ^{t}$, he takes in to account not only the consequences of his decision
for the future behaviour of the monopolist, but also the effect on the
behaviour of his own future selves. Similarly, when the durable goods
monopolist chooses the current price $p^{t}$, he takes into account the
implications for current profits, future regulation, as well as the
behaviour of his future selves.

Again, we focus on the Markov perfect equilibrium of the taxation game. This
will enable us to compare our results to the benchmarks studied in the
previous chapter. The period-$t$ strategies of the regulator and the
monopolist are restricted to be functions of the two payoff-relevant state
variables:

\begin{enumerate}
\item the previous period's price $p^{t-1}$; and,

\item the previous period's tax level $\tau ^{t-1}$.
\end{enumerate}

Let the strategies of the regulator and the monopolist be $\tau
^{t}=f(p^{t-1},\tau ^{t-1})$ and $p^{t}=g(p^{t-1},\tau ^{t-1})$,
respectively. We assume that these functions are continuously
differentiable. This allows us to use dynamic programming to characterise
them and rules out a potential multiplicity of equilibria.

Also, we need to specify how players form their expectations about future
prices. Just as before, we assume that agents have perfect rational
expectations: they correctly anticipate future prices both on and off the
equilibrium path. Given our focus on Markov strategies, this assumption
implies that $p_{e}^{t+1}=g(p^{t},\tau ^{t})$.

To solve for the equilibrium strategies, we formulate the problems of the
monopolist and the regulator recursively. The solution concept of Markov
perfect equilibrium requires that these strategies solve a pair of Bellman
equations. The regulator's Bellman equation states that the equilibrium tax
rate must maximise the net present value of welfare: 
\begin{multline}
W(p^{t-1},\tau ^{t-1})  \label{tax:Bell_reg} \\
=\max_{\tau ^{t}}\Bigg\{w\bigg(p^{t-1},g\big(p^{t-1},\tau ^{t-1}\big),g\Big(g%
\big(p^{t-1},\tau ^{t-1}\big),\tau ^{t},\tau ^{t-1}\Big),\tau ^{t}\bigg) \\
+\delta W\Big(g\big(p^{t-1},\tau ^{t-1}\big),\tau ^{t}\Big)\Bigg\}.
\end{multline}%
The monopolist's Bellman equation states the equilibrium pricing strategy
must maximise the discounted stream of profits: 
\begin{multline}
\Pi \big(p^{t-1},\tau ^{t-1}\big)=\max_{p^{t}}\Bigg\{\pi \bigg(%
p^{t-1},p^{t},g\Big(p^{t},f\big(p^{t-1},\tau ^{t-1}\big)\Big),f\big(%
p^{t-1},\tau ^{t-1}\big)\bigg)  \label{tax:Bell_mon} \\
+\delta \Pi \bigg(p^{t},f\big(p^{t-1},\tau ^{t-1}\big)\bigg)\Bigg\}.
\end{multline}%
Moreover, time invariance of Markov-perfect strategies requires that 
\begin{multline}
f(p^{t-1},\tau ^{t-1})  \label{tax:strat_reg} \\
=\arg \max_{\tau ^{t}}\Bigg\{w\bigg(p^{t-1},g\big(p^{t-1},\tau ^{t-1}\big),g%
\Big(g\big(p^{t-1},\tau ^{t-1}\big),\tau ^{t},\tau ^{t-1}\Big),\tau ^{t}%
\bigg) \\
+\delta W\Big(g\big(p^{t-1},\tau ^{t-1}\big),\tau ^{t}\Big)\Bigg\},
\end{multline}%
and 
\begin{multline}
g(p^{t-1},\tau ^{t-1})=\arg \max_{p^{t}}\Bigg\{\pi \bigg(p^{t-1},p^{t},g\Big(%
p^{t},f\big(p^{t-1},\tau ^{t-1}\big)\Big),f\big(p^{t-1},\tau ^{t-1}\big)%
\bigg)  \label{tax:strat_mon} \\
+\delta \Pi \bigg(p^{t},f\big(p^{t-1},\tau ^{t-1}\big)\bigg)\Bigg\}.
\end{multline}

The recursive formulation yields strategies that prescribe optimal actions
for any values of the state variables. Thus, optimality is ensured for any
history of play. Consequently, the Markov-perfect equilibrium is also
subgame perfect, and therefore time consistent.

\subsection{Resolving the regulator's time inconsistency}

\label{sec:time-cons-strat}

An inspection of the regulator's Bellman equation \eqref{tax:Bell_reg}
reveals that its right hand side no longer depends on the regulator's future
decision, $\tau ^{t+1}$, but only on his current strategy, $\tau ^{t}$, and
the current state, $(p^{t-1},\tau ^{t-1})$. Thus, delegation resolves the
regulator's time consistency problem by effectively removing the jump
variable from instantaneous welfare. The economic interpretation is that the
regulator's period-$(t+1)$ decision no longer imposes an externality on
period-$t$ welfare.

In our setting, taxation serves as an intrapersonal commitment device: the
current regulator is unable to interfere in the choice of the current price.
Note that the social planner is free to affect future prices. However, the
period $t$ and $t+1$ regulators do not disagree about the choice of $p^{t+2}$%
. Thus, if the cost of delegation is low, they will provide the monopolist
with incentives to choose the first-best (i.e. precommitment)  price path.

The above observation is crucially dependent upon the assumption of
simultaneous choice of prices and taxes. If the monopolist knew the tax rate
before he set his price level, the current regulator would be able to
influence the monopolist's current price choice. This would create a
temptation for the regulator to deviate from the plan preferred by his
predecessor. Analytically, there would still be a jump variable in
instantaneous welfare.

\subsection{Equilibrium strategies}

\label{sec:equil-strat}

Bellman equations \eqref{tax:Bell_reg}--\eqref{tax:strat_mon} yield a pair
of Euler equations which characterise the equilibrium strategies of the
monopolist and the regulator. They are provided in the following
proposition. 
\begin{prop}
  Suppose that Assumptions \ref{dur:ass1}, \ref{dur:ass2} and
  \ref{dur:ass3} are satisfied. The Markov perfect equilibrium
  strategies solving the taxation game from period $2$ onward satisfy
  the necessary conditions defined by the following generalised
  Euler-Lagrange equations:
  \begin{description}
  \item[Regulator's condition]
    \begin{equation} \label{reg:euler} w^t_1 - g^t_1 \left[
        \frac{\delta w^t_4 + w^{t-1}_5}{\delta g^t_2} \right] +
      \frac{\delta w^{t-1}_2 + w^t_3}{\delta^2} + \frac{\delta
        w^{t-1}_4 + w^{t-2}_5}{\delta^2 g^{t-1}_2} = 0 \quad \forall
      \quad t \geq 0
    \end{equation}
  \item[Monopolist's condition]
    \begin{multline} \label{mon:euler} \pi^t_3 g^{t+1}_2 + \pi^t_4 +
      \frac{\pi^{t-1}_2 + \pi^{t-1}_3 g^t_1 + \delta\pi^t_1}{\delta
        f^t_1} \\ = \delta f^{t+1}_2 \left[ \pi^{t+1}_3 g^{t+2}_2 +
        \frac{\pi^t_2 + \pi^t_3 g^{t+1}_1 + \delta\pi^{t+1}_1}{\delta
          f^{t+1}_1} \right] \quad \forall \quad t \geq 0.
    \end{multline}
  \end{description}
\end{prop}

\begin{proof}
See Appendix \ref{cha:equil-durable-goods}.
\end{proof}

\section{Comparison with benchmarks}

\label{sec:comp-with-benchm}

If we substitute the derivatives of \eqref{eq:19}, we can rewrite the
regulator's Euler equation as%
\begin{multline}
\pi _{3}^{t-1}+CS_{3}^{t-1}-\psi _{3}^{t-1}+(\alpha -1)\tau ^{t-1}\psi
_{3}^{t-1}  \label{eq:20} \\
+\frac{1}{g_{2}^{t}}\bigg[(\alpha -1)\psi ^{t-1}+2\theta \Big(\delta \big( %
\tau ^{t}-\tau ^{t-1}\big)-\big(\tau ^{t-1}-\tau ^{t-2}\big)\Big)\bigg] \\
+\delta \big(\pi _{2}^{t}-\psi _{2}^{t}+CS_{2}^{t}+(\alpha -1)\tau ^{t}\psi
_{2}^{t}\big)+\delta ^{2}\big(\pi _{1}^{t+1}-\psi
_{1}^{t+1}+CS_{1}^{t+1}+(\alpha -1)\tau ^{t+1}\psi _{1}^{t+1}\big) \\
-\frac{\delta g_{1}^{t+1}}{g_{2}^{t+1}}\Big[(\alpha -1)\psi ^{t}-2\theta %
\big(\tau ^{t}-\tau ^{t-1}\big)+2\delta \theta \big(\tau ^{t+1}-\tau ^{t} %
\big)\Big]=0.
\end{multline}
When $\theta =0$ and $\alpha =1$, this equation replicates the precommitment
Euler equation, \eqref{five}. The implication is that if policy adjustment
costs are zero and the government's valuation of the tax revenue is equal to
the firm's valuation of the cost of taxation, then the regulator will choose
the tax rate in such a way as to induce the monopolist to follow the
first-best price path.

Our taxation mechanism provides the regulator with commitment opportunities
for one period at a time. It enables him to overcome his time inconsistency.
As long as his incentives are not distorted by other considerations, he can
achieve the first best outcome. In chapter \ref{cha:computation} we explore
the extent to which the price is distorted by revenue valuation and costs of
changing policy.

%%% Local Variables:
%%% mode: latex
%%% TeX-master: "root"
%%% End:

% LocalWords:  precommitment precommit duopolists duopolist Stackelburg
