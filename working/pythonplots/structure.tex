\documentclass{amsart}

\title{Structure of Python programme for data analysis and graph construction}
\date{\today}

\begin{document}
\maketitle
\section{Objectives}
\label{sec:objectives}

\begin{enumerate}
\item Generate convergence paths from arbitrary starting points across
  multiple parameter values
\item Generate welfare, consumer surplus, profit and pollution paths
  to correspond with the price and tax paths
\item Create plots of the above. Both scatterplots of the steady
  states and time series plots of convergence.
\end{enumerate}

\section{Tasks}
\label{sec:tasks}

\begin{enumerate}
\item Define the necessary functions:
\item
  \begin{enumerate}
  \item Strategy functions
  \item Profit function
  \item Pollution function
  \item Consumer surplus function
  \item Welfare function
  \end{enumerate}
\item Initiate the parameter variables
\item Read in the data and populate the parameter variables
\item Generate data matrices from the functions defined above
\item Plot the steady states
\item Plot the convergence of each function
\end{enumerate}

\section{Programme structure}
\label{sec:programme-structure}

\subsection{Files}
\label{sec:files}

\begin{itemize}
\item \texttt{jump\_pydata.csv} contains the Maple results in an
  appropriate matrix
\item \texttt{jump\_funcs.py} defines the necessary functions to
  characterise the game
\item \texttt{jump\_data\_handler.py} reads in the data from
  \texttt{jump\_pydata.csv} file and generates further matrices using
  the functions from \texttt{jump\_funcs.py}. The entire data set is
  then written to a Numpy array for easy import in future
\item \texttt{jump\_plot.py} reads in the Numpy array created by
  \texttt{jump\_data\_handler.py} and generates the scatter and times
  series plots
\end{itemize}

\subsection{Functions}
\label{sec:functions}

\begin{description}
\item[Strategy functions]
  \begin{align}
    \label{eq:1}
    f(p^{t-1},\tau^{t-1}) &= \alpha_r + \gamma_{r1} p^{t-1} +
    \gamma_{r2} \tau^{t-1}, \\
    g(p^{t-1},\tau^{t-1}) &= \alpha_m + \gamma_{m1} p^{t-1} +
    \gamma_{m2} \tau^{t-1}.
  \end{align}
\item[Demand function]
  \begin{equation}
    \label{eq:2}
      x^t = p^{t-1} - (\beta+1)p^t + \beta p^{t+1}.
  \end{equation}
\item[Production cost]

\item[Pollution function]

\item[Profit function]


\end{description}


\end{document}
