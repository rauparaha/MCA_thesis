\documentclass{amsart}

\usepackage{amssymb,amsmath}

\newcommand{\del}[2]{\frac{\partial #1}{\partial #2}}
\newcommand{\ddel}[2]{\frac{\partial^2 #1}{\partial #2^2}}
\renewcommand{\d}[2]{\frac{d #1}{d #2}}

\linespread{1.5}

\title{Hyperbolic discounting model: \\General functional forms} \date{\today}


\begin{document}

\maketitle

\section{Elements of the model}
\label{sec:elements-model}

\subsection{Consumer}
\label{sec:consumer}

Imagine a representative consumer with quasi-linear preferences across
two goods: a polluting good, $x$, and a numeraire, $m$. The consumer's
instantaneous utility function is
\begin{equation}
  \label{eq:1}
  U(x,m) = u(x) + m.
\end{equation}
The function $u(x)$ must satisfy the Inada conditions in order for the
inverse demand to be defined over all possible values of
$x$.\textbf{[does this need a proof?]}

The consumer's problem is then
\begin{gather}
  \label{eq:2}
  \max_x u(x) + m \\
  \mbox{s.t. } px + m \leq I.
\end{gather}
The FOC of this problem is
\begin{equation}
  \label{eq:3}
  u'(x) = p
\end{equation}
so inverse demand in this market is
\begin{equation}
  \label{eq:4}
  p=u'(x).
\end{equation}

\subsection{Monopolist}
\label{sec:monopolist}

The manufacturer of good $x$ faces costs $C(x)$ and monopolises the
market. His instantaneous profit function is
\begin{align}
  \label{eq:5}
  \pi &= R(x) - C(x) \\
  &= u'(x).x - C(x).
\end{align}


\subsection{Regulator}
\label{sec:regulator}

The regulator recognises the pollution damage, $\phi(k_t)$, caused by
the existence of the stock, $k_t$, of good $x$. The stock of the good
evolves according to the rule
\begin{equation}
  \label{eq:6} k_t = \theta k^{t-1} + x^{t-1} \qquad \theta \in [0,1].
\end{equation}

Instantaneous welfare is
\begin{align}
  \label{eq:7}
  w(x_t,k_t) &= \pi(x_t) + CS(x_t) - \phi(k_t) \\
  &=  u'(x_t).x_t - C(x_t) + \int_0^{x_t} u'(x)\; dx - u'(x_t).x_t - \phi(k_t) \\
  &= u(x_t) - C(x_t) - \phi(k_t)
\end{align}

The regulator suffers from time inconsistency and is modelled as
having quasi-hyperbolic preferences with $\beta\delta$-discounting. As
a result, the regulator's net present valuation of welfare is
\begin{equation}
  \label{eq:8}
  W(x,k) = w(x_t,k_t) + \beta \sum_{i=1}^\infty \delta^i w(x_{t+i},k_{t+i}).
\end{equation}

\section{Laissez Faire equilibrium}
\label{sec:laiss-faire-equil}

Suppose that the monopolist acted unregulated. He then solves a static
problem in each period:
\begin{equation}
  \label{eq:9}
  \max_x u'(x).x - C(x).
\end{equation}
The FOC of this problem is
\begin{equation}
  \label{eq:10}
  u''\left( x^{\ell}\right) .x^{\ell} + u'\left( x^{\ell}\right)  - C'\left( x^{\ell}\right)  = 0
\end{equation}
where $x^{\ell}$ denotes the laissez-faire level of output chosen by the
monopolist. This shows the point at which marginal profit is equal to
zero.


\section{Regulation with precommitment}
\label{sec:regul-with-prec}

Suppose that the regulator can directly choose $\{ x_\tau
\}_{\tau=t}^\infty$ at time $t$. The regulator must maximise
\begin{equation}
  \label{eq:11}
  W_t(x,k) = \beta \sum_{i=1}^\infty \delta^i \left[ u(x_{t+i}) -
    C(x_{t+i}) - \phi(k_{t+i}) \right] + \left[ u(x_t) -
    C(x_t)  - \phi(k_t) \right]
\end{equation}
where $\delta$ is the discount rate and $\beta$ is the
quasi-hyperbolic modifier on the future discount rate.

This can be notated more simply as
\begin{equation}
  \label{eq:12} W_0 = w(x_0,k_0) + \beta\delta w(x_1,\theta k_0+x_0) +
\beta\delta^2 w(x_2,x_1+\theta x_0+\theta^2 k_0) + \ldots
\end{equation} Maximising this requires taking first-order conditionss
with respect to $x_t \forall t\in [0,\infty)$. Derivatives are notated
as usual with the time superscript taken from the time subscript of
the state variable.
\begin{align}
  \label{eq:13} \del{W_0}{x_0} &= w_1^0 + \beta\delta w_2^1 +
\theta\beta\delta^2 w_2^2 + \ldots = 0 \\ \label{eq:14} \del{W_0}{x_1}
&=\beta\delta w_1^1 + \theta\beta\delta^2 w_2^2 + \theta\beta\delta^3
w_2^3 + \ldots = 0 \\ \label{eq:15} \del{W_0}{x_2} &=\beta\delta^2
w_1^2 + \theta\beta\delta^3 w_2^3 + \theta\beta\delta^4 w_2^4 + \ldots
= 0 \\ &\hspace{2cm}\vdots
\end{align} Compacting these conditions,
\eqref{eq:13}$-$\eqref{eq:14}$\times\theta$ gives
\begin{equation}
  \label{eq:16} w^0_1 + \beta\delta \left(w^1_2 - \theta w^1_1 \right)
= 0,
\end{equation} and $($\eqref{eq:15}$-$\eqref{eq:16}$\times\theta ) /
\beta\delta$ gives
\begin{equation}
  \label{eq:17} w^t_1 + \delta \left(w^{t+1}_2 - \theta w^{t+1}_1 \right)
= 0 \qquad \forall\; t \geq 2
\end{equation}
which expands to
\begin{equation}
  \label{eq:18}
  u'(x_t) - C'(x_t) - \delta\theta \left( u'(x_{t+1}) -
    C'(x_{t+1})\right) - \delta\phi'(k_{t+1}) = 0 \qquad \forall\; t
  \geq 2
\end{equation}

\subsection{Comparison to laissez-faire outcome}
\label{sec:comp-laiss-faire}

At this point it is important to see how the first-best outcome
compares to the laissez-faire outcome previously derived
(equation ~\eqref{eq:10}). The laissez faire equilibrium is
characterised by equation \eqref{eq:10}:
\begin{equation}
  \label{eq:19}
  u'\left( x^{\ell}\right) - C'\left( x^{\ell}\right) + u''\left(
    x^{\ell}\right) x^{\ell} = 0.
\end{equation}
This equates the monopolist's marginal profit to zero. Since profit is
concave, marginal profit is a decreasing function.

In order to tractably compare this equilibrium condition to the first
best outcome it is necessary to focus upon the steady state of the
dynamic game. Suppose that $x_1 = x_2 = \ldots = x_T = \bar{x}$, and
similarly for the state variable, $k_t$. Now rearrange equation
\eqref{eq:18}
\begin{equation}
  \label{eq:20}
  u'\left( \bar{x}\right) - C'\left( \bar{x}\right) =
  \frac{\delta\phi'\left( \bar{k}\right)}{1-\delta\theta}
\end{equation}
and add $u''\left( \bar{x}\right) \bar{x}$ to both sides:
\begin{equation}
  \label{eq:21} 
  u'\left( \bar{x}\right) - C'\left( \bar{x}\right) +
  u''\left( \bar{x}\right) \bar{x} = \frac{\delta\phi'\left(
      \bar{k}\right)}{1-\delta\theta} + u''\left( \bar{x}\right)
  \bar{x}. 
\end{equation}
Remembering that marginal profit is a decreasing function, if
$\frac{\delta\phi'\left( \bar{k}\right)}{1-\delta\theta} + u''\left(
  \bar{x}\right) \bar{x} > 0$ then $\bar{x} < x^{\ell}$ and vice
versa.

Signing the component parts gives
\begin{align}
  \label{eq:22}
  \delta\phi'\left( \bar{x}\right) &> 0 \\
  1-\beta\delta &> 0 \\
  \intertext{by assumption, and} 
  u''\left( \bar{x}\right) \bar{x} &< 0
\end{align}
because $u(\cdot)$ exhibits diminishing marginal utility. So if
\begin{equation}
  \label{eq:23}
  \frac{\delta\phi'\left( \bar{k}\right)}{1-\delta\theta} > u''\left(
    \bar{x}\right) \bar{x}
\end{equation}
then $\bar{x} < x^{\ell}$. The left hand component of the inequality shows the
cost of the externality, while the right hand component shows the
deadweight loss due to the exercise of monopoly power. The externality
effect causes the quantity produced to be too high, while the firm's
market power depresses production. Regulation to diminish pollution is
only worthwhile when the former outweighs the latter. Henceforth, we
shall assume that equation \eqref{eq:23} holds.


\section{Regulation without precommitment}
\label{sec:regul-with-prec-2}

Now suppose that the regulator can still directly choose output but is
no longer able to precommit to future output decisions. The problem
must be formulated recursively in order to solve for a time consistent
output path.

The problem will first be solved with reduced form notation and then
functional forms can later be substituted in.

Let the MPE strategy of the regulator be $x_t = f(k_t)$. Then his
current period value function is
\begin{equation}
  \label{eq:24} U(k_t) = \max_{x_t} \left\{ w(x_t,k_t) + \beta\delta
V( \theta k_t + x_t) \right\}.
\end{equation} From period $t+1$ onward the value function is
$V(k_t)$.
\begin{equation}
  \label{eq:25} V(k_t) = w\left( f(k_t),k_t\right) + \delta V\left(
\theta k_t + f(k_t) \right).
\end{equation}

The current period's FOC is
\begin{align} w^t_1 + \beta\delta V^{t+1}_1 &= 0 \\ \label{eq:26}
\therefore V^{t+1}_1 &= -\frac{w^t_1}{\beta\delta}.
\end{align} Differentiating \eqref{eq:25} gives
\begin{equation}
  \label{eq:27} V^t_1 = w^t_1f^t_1 + w^t_2 + \delta V^{t+1}_1 (\theta
f^t_1)
\end{equation} and substituting in equation \eqref{eq:26} gives the
reduced form Euler-Lagrange equation:
\begin{equation}
  \label{eq:28} w^t_1 + \beta\delta( w^{t+1}_1 f^{t+1}_1 + w^{t+1}_2 )
- \delta (\theta + f^{t+1}_1)w^{t+1}_1 = 0.
\end{equation}

This expands to give
\begin{multline}
  \label{eq:29}
  u'(x_t) - C'(x_t) + \beta\delta\left( \left(
      u'(x_{t+1}) - C'(x_{t+1}) \right) f^{t+1}_1 - \phi'(k_t) \right)\\
  - \delta (\theta + f^{t+1}_1)\left( u'(x_{t+1}) - C'(x_{t+1})
  \right) = 0.
\end{multline}

It is unnecessary to compare this outcome directly to the first best
\textbf{[I hope.]} Rather note that it is significantly different from
the first best outcome's generalised Euler equation and it will thus
produce a different price path. \textbf{[This is totally inadequate.
  Is this even true?!]}


\section{Regulation with delegation}
\label{sec:regul-with-deleg}

The delegation game involves the regulator setting a tax rate for
pollution simultaneously with the monopolist's choice of output in
each period. Both the tax and the choice of output are feedback
strategies. We consider only a linear tax, however there are two
possible ways to levy it. It can be levied on either emissions or upon
the stock of pollution.

The case of a tax on emissions will be considered first and then the
problem with a pollution tax will be solved.

\subsection{The welfare function}
\label{sec:welfare-function}

The regulator's problem changes for two reasons: first, because he
gains revenue from taxation and, secondly, because we introduce a cost
to changing the tax rate. Economists are often criticised by
policymakers for excluding the costs of implementation when they
recommend taxes. Here, we explicitly include the costs of implementing
and modifying tax schemes in the regulator's welfare function.

Suppose that the tax is levied on emissions, the value of the tax
revenue to the regulator is $\gamma$, and the the cost of changing
policies is $\kappa \rho(\tau_t,\tau_{t-1})$ \textbf{[would
  $\rho\left(\left| \tau_t - \tau_{t-1}\right|\right)$ be more
  appropriate and accurate?]}, where $\tau_t$ is the period $t$ tax
rate. Then the welfare function becomes
\begin{equation}
  \label{eq:30}
  w(x_t,k_t,\tau_t,\tau_{t-1}) = u(x_t) - C(x_t) - \phi(k_t) +
  (\gamma -1)\tau_tx_t - \kappa \rho(\tau_t,\tau_{t-1}).
\end{equation}
\textbf{[what conditions should I impose on $\rho$]}
Note that, if $\gamma=1$ then the tax is a simple transfer
from the monopolist to the consumers and doesn't directly affect the
welfare function. Similarly, if $\kappa=0$ then there are no costs of
policy adjustment.

\subsection{The profit function}
\label{sec:profit-function}

With taxation the monopolist's instantaneous profit becomes
\begin{equation}
  \label{eq:31}
  u'(x_t).x_t - C(x_t) - \tau_tx_t.
\end{equation}


\subsection{The game}
\label{sec:game}

Let the MPE strategy of the monopolist be $x_t = h(\tau_{t-1},k_t)$
and the MPE strategy of the regulator be $\tau_t = g(\tau_{t-1},k_t)$.

\subsubsection{Regulator}
\label{sec:regulator-1}

Current period value function:
\begin{equation}
  \label{eq:32}
  U(\tau_{t-1},k_t) = \max_{\tau_t} \bigg\{ w\big(
    h(\tau_{t-1},k_t),k_t,\tau_t,\tau_{t-1}\big) + \beta\delta V
  \big( \tau_t,\theta k_t + h(\tau_{t-1},k_t) \big) \bigg\}
\end{equation}

Continuation value function:
\begin{multline}
  \label{eq:33}
  V(\tau_{t-1},k_t) = w\big(
  h(\tau_{t-1},k_t),k_t,g(\tau_{t-1},k_t),\tau_{t-1}\big) \\ +
  \delta V \big( g(\tau_{t-1},k_t),\theta k_t + h(\tau_{t-1},k_t)
  \big)
\end{multline}

FOC:
\begin{align}
  \label{eq:34}
  w^t_3 + \beta\delta V^{t+1}_1  &= 0 \\
  \label{eq:35} \therefore V^{t+1}_1 &= -\frac{w^t_3}{\beta\delta}
\end{align}

Envelope conditions:
\begin{align}
  \label{eq:36}
  V^t_1 &= w^t_1h^t_1 + w^t_3g^t_1 + w^t_4 + \delta \left[
V^{t+1}_1g^t_1 + V^{t+1}_2h^t_1 \right] \\ \label{eq:37}
  V^t_2 &= w^t_1h^t_2 + w^t_2 + w^t_3g^t_2 + \delta \left[
    V^{t+1}_1g^t_2 + V^{t+1}_2 \left( \theta + h^t_2 \right)\right]
\end{align}

Now substituting \eqref{eq:35} into \eqref{eq:36} gives
\begin{gather}
  \label{eq:38}
  w^t_1h^t_1 + w^t_3g^t_1 + w^t_4 - \frac{w^t_3g^t_1}{\beta} + \delta
  V^{t+1}_2h^t_1 + \frac{w^{t-1}_3}{\beta\delta} = 0 \\ \label{eq:39}
  \therefore V^{t+1}_2 = \frac{w^t_3g^t_1}{\beta\delta h^t_1} -
  \frac{w^t_1h^t_1 + w^t_3g^t_1 + w^t_4}{\delta h^t_1} -
  \frac{w^{t-1}_3}{\beta\delta^2 h^t_1}
\end{gather}
and \eqref{eq:39},~\eqref{eq:36}$\rightarrow$~\eqref{eq:37}:
\begin{multline}
  \label{eq:40}
  \frac{w^{t-1}_3g^{t-1}_1 - \beta\left( w^{t-1}_1h^{t-1}_1 +
      w^{t-1}_3g^{t-1}_1 + w^{t-1}_4 \right)}{\delta h^{t-1}_1} \\ -
  \frac{w^{t-2}_3}{\delta^2 h^{t-1}_1} =
  \beta ( w^t_1h^t_2 + w^t_2 + w^t_3g^t_2 ) - g^t_2w^t_3 \\
  +\frac{\left( \theta + h^t_2 \right)}{h^t_1} \left[ w^t_3g^t_1 -
    \beta\left( w^t_1h^t_1 + w^t_3g^t_1 + w^t_4 \right) -
    \frac{w^{t-1}_3}{\delta} \right].
\end{multline}

In the special case where $\gamma=1, \kappa=0$ this can be simplified
by substituting in the partial derivatives of the welfare function:
\begin{align}
  \label{eq:41}
  w^t_1 &= u'(x_t) - c'(x_t) \\ \label{eq:42} w^t_2 &= -\phi'(x_t)
  \\ \label{eq:43} w^t_3 &= 0 \\ \label{eq:44} w^t_4 &= 0.
\end{align}

After substitution the Euler equation, \eqref{eq:40} reduces to
\begin{equation}
  \label{eq:45}
  \frac{u'(x_{t-1}) - c'(x_{t-1})}{\delta} - \phi'(x_t) - \theta \big(
  u'(x_t) - c'(x_t) \big) = 0.
\end{equation}
Multiplying by $\delta$ and shifting the equation forward one period
gives
\begin{equation}
  \label{eq:46}
  u'(x_t) - c'(x_t) - \delta \phi'(x_{t+1}) - \delta\theta \big(
  u'(x_{t+1}) - c'(x_{t+1}) \big) = 0
\end{equation}
which replicates the result of equation \eqref{eq:18}, the
precommitment outcome.

\subsubsection{Monopolist}
\label{sec:monopolist-1}

Since the monopolist discounts exponentially, he has a stationary
value function:
\begin{equation}
  \label{eq:47}
  \Pi(\tau_{t-1},k_t) = \max_{x_t} \left\{
    \pi(x_t,g\left(\tau_{t-1},k_t)\right) + \delta \Pi \left(
      g(\tau_{t-1},k_t), \theta k_t + x_t \right) \right\}.
\end{equation}`
FOC:
\begin{align}
  \pi^t_1 + \delta \Pi^{t+1}_2 &= 0 \\  \label{eq:48}
  \therefore \Pi^{t+1}_2 &= -\frac{\pi^t_1}\delta.
\end{align}
Envelope conditions:
\begin{align}
  \label{eq:49}
  \Pi^t_1 &= \pi^t_2g^t_1 + \delta \Pi^{t+1}_1g^t_1 \\\label{eq:50}
  \Pi^t_2 &= \pi^t_2g^t_2 + \delta \Pi^{t+1}_1g^t_2 + \theta\delta\Pi^{t+1}_2.
\end{align}
Solving for the Euler equation:
\eqref{eq:48}$\to$\eqref{eq:50} gives
\begin{align}
  \label{eq:51}
  -\frac{\pi^{t-1}_1}\delta &= \pi^t_2g^t_2 - \theta\pi^t_1 + \delta
  \Pi^{t+1}_1g^t_2 \\\label{eq:52}
  \Pi^{t+1}_1 &= \frac{\theta\pi^t_1 - \pi^t_2g^t_2 -
    \frac{1}\delta\pi^{t-1}_1}{\delta g^t_2}.
\end{align}
Now, (\ref{eq:52})$\rightarrow$(\ref{eq:49}) yields the Euler-Lagrange
equation:
\begin{equation}
  \label{eq:53}
  \pi^t_2g^t_1 + \frac{g^t_1}{g^t_2} \left( \theta \pi^t_1 -
    \pi^t_2g^t_2 - \frac{1}\delta\pi^{t-1}_1 \right) -
  \frac{\theta\pi^{t-1}_1 - \pi^{t-1}_2g^{t-1}_2 -
    \frac{1}\delta\pi^{t-2}_1}{\delta g^{t-1}_2} = 0
\end{equation}

The derivatives of the profit function can now be found and
substituted in.
\begin{multline}
  \label{eq:54}
  (-x_t)g^t_1 + \frac{g^t_1}{g^t_2} \bigg( \theta \bigg( u''(x_t)x_t +
      u'(x_t) - C'(x_t) - \tau_t \bigg) \\ -
    (-x_t)g^t_2 - \frac{1}\delta\bigg( u''(x_{t-1})x_{t-1} +
      u'(x_{t-1}) - C'(x_{t-1}) - \tau_{t-1} \bigg) \bigg) \\ - 
  \frac{\theta\bigg( u''(x_{t-1})x_{t-1} + u'(x_{t-1}) - C'(x_{t-1}) - \tau_{t-1} \bigg) - (-x_{t-1})g^{t-1}_2 \\-
    \frac{1}\delta\bigg( u''(x_{t-2})x_{t-2} + u'(x_{t-2}) - C'(x_{t-2}) - \tau_{t-2} \bigg)}{\delta g^{t-1}_2} = 0 
\end{multline}







\end{document}

%%% Local Variables:
%%% mode: latex
%%% TeX-master: t
%%% End:

% LocalWords:  externality deadweight
