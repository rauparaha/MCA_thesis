\documentclass{beamer}
%\documentclass{article}
%\usepackage{beamerarticle}

\mode<presentation> { \usetheme{Warsaw}
  \usefonttheme[onlysmall]{structurebold} \usecolortheme{whale}
  \setbeamercovered{transparent} }


%\usepackage[english]{babel} or whatever

\usepackage[latin1]{inputenc}
\usepackage{graphicx} \usepackage{times} \usepackage[T1]{fontenc}
% Or whatever. Note that the encoding and the font should match. If T1
% does not look nice, try deleting the line with the fontenc.

\newcommand{\bm}[1]{\mathbf{#1}}

\title[Taxing for time consistency]{Slaying the DI dragon}

\subtitle{Taxation mechanisms as precommitment devices.}
\author[Zuccollo]{James Zuccollo}

\institute[VUW]{Victoria University of Wellington}

\date[Micro workshop]{Micro workshop, June 2008}
%\subject{Microeconomics}

% If you have a file called "university-logo-filename.xxx", where xxx
% is a graphic format that can be processed by latex or pdflatex,
% resp., then you can add a logo as follows:

% \pgfdeclareimage[height=0.5cm]{university-logo}{university-logo-filename}
% \logo{\pgfuseimage{university-logo}}



% Delete this, if you do not want the table of contents to pop up at
% the beginning of each subsection: \AtBeginSubsection[] {
%   \begin{frame}<beamer> \frametitle{Outline}
%     \tableofcontents[currentsection,currentsubsection]
%   \end{frame}
% }


% If you wish to uncover everything in a step-wise fashion, uncomment
% the following command:

% \beamerdefaultoverlayspecification{<+->}


\begin{document}

\begin{frame}
  \titlepage
\end{frame}

\begin{frame}
  \frametitle{Outline}
  \tableofcontents[pausesections]
  % You might wish to add the option [pausesections]
\end{frame}


% Structuring a talk is a difficult task and the following structure
% may not be suitable. Here are some rules that apply for this
% solution:

% - Exactly two or three sections (other than the summary).  - At
% *most* three subsections per section.  - Talk about 30s to 2min per
% frame. So there should be between about 15 and 30 frames, all told.

% - A conference audience is likely to know very little of what you
% are going to talk about. So *simplify*!  - In a 20min talk, getting
% the main ideas across is hard enough. Leave out details, even if it
% means being less precise than you think necessary.  - If you omit
% details that are vital to the proof/implementation, just say so
% once. Everybody will be happy with that.

\begin{frame}
  \frametitle{Aims}
  We attempt to fix two problems with one tax and no rule of law!
\begin{block}{The context}
  \begin{itemize}
  \item<1-> Optimal taxation of polluters is a time honoured topic.
  \item<1-> Overcoming dynamic inconsistency is less well known but
    has been tackled before.
  \item<2-> Here we attempt to tackle both problems simultaneously
    through a single taxation mechanism.
  \end{itemize}
\end{block}
\end{frame}

\begin{frame}
  \frametitle{Questions} 
  \begin{enumerate}
    \item<1-> How does dynamic inconsistency affect the regulation of polluters?
    \item<2-> Is there a mechanism that can overcome the
      inefficiencies of both dynamic inconsistency and pollution simultaneously?
    \item<3-> What are the limitations of the mechanism?
    \end{enumerate}
  \end{frame}

\section{Background}

\subsection{What are we tackling?}

\begin{frame}
  \frametitle{Setting the scene}
  \begin{itemize}
  \item<1-> A regulator is faced with the problem of correcting the
    externality created by a polluting firm.
  \item<2-> The regulator suffers from dynamic inconsistency and is
    unable to commit to future actions.
  \item<3-> Depending upon the cause of the time inconsistency, the
    regulator can be modelled in either of two ways.
  \item<4-> Once the regulator's problem has been modelled we can look
    for a taxation scheme that gets as close to the first-best outcome
    as possible.
  \end{itemize}
\end{frame}

\begin{frame}
  \frametitle{Dynamic inconsistency}
  \begin{block}{Definition}
    A player in a dynamic game is dynamically inconsistent if he
    will, in future, wish to deviate from a currently optimal plan of
    action.
  \end{block}\pause
  There are two causes of dynamic inconsistency:
  \begin{itemize}
  \item Jump states; and,
  \item Hyperbolic discounting.
  \end{itemize}
\end{frame}

\begin{frame}
  \frametitle{Jump states}
  \begin{block}{Definition}
    A jump state exists when a player's current instantaneous payoff
    depends on his expected future actions.
  \end{block}
  For example:
  \begin{itemize}
  \item<1-> Inflation expectations influence current wages;
  \item<2-> Expected prices influence current mineral extraction rates.
  \end{itemize}
\end{frame}

\begin{frame}
  \frametitle{Consequences of jump states}
If your current payoff depends on future actions then
  \begin{itemize}
  \item<1-> you predict the future action based on your currently
    preferred future action,
  \item<2-> you get to the future and have to decide what to do,
  \item<3-> but you no longer care about the effect on your past
    payoff because that's sunk,
  \item<4-> so your preferred action differs from what you expected it
    to be, since your objective function for the period has changed,
  \item<5-> which means you won't maximise your lifetime utility
    because you keep changing your mind.
  \end{itemize}
\end{frame}

\begin{frame}
  \frametitle{Hyperbolic discounting}
  \begin{block}{Definition}
    Hyperbolic discounting describes a situation in which preferences
    are non-stationary because the discount rate is not constant over
    time.
  \end{block}
  \begin{itemize}
  \item<1-> Empirical research suggests that peoples' discount rate
    declines as he consider decisions further in the future
  \item<2-> Hence, the discounted sum of lifetime utility for a
    sequence of future actions depends on the time period that the
    actions are viewed from, and decisions are not consistent over
    time.
  \end{itemize}
\end{frame}

\subsection{Inconsistency in the 'real' world}

\begin{frame}
  \frametitle{Situations which give rise to inconsistency}
\begin{itemize}
\item<1-> Jump states commonly arise in the modelling of:
  \begin{itemize}
  \item<1-> Durable goods producers
  \item<2-> Addictive goods producers
  \item<3-> Exhaustible resource extraction.
  \end{itemize}
\item<4-> Hyperbolic discounting is used in many contexts since it is
  presented as a descriptive model of normal human behaviour.
\item<5-> Our regulator could suffer from time inconsistency due to
  either of these factors and both would inhibit him from efficiently
  controlling pollution.
\end{itemize}
\end{frame}

\begin{frame}
  \frametitle{Implications for regulators}
  \begin{itemize}
  \item<1-> A regulator who encounters jump states, or whose
    preferences exhibit hyperbolic discounting, may not be able to
    achieve the first best action path.
  \item<2-> In our context that might result in under-regulation of
    pollution, which could have irreversible consequences for the
    environment.
  \item<3-> We develop an illustrative model for each type of
    inconsistency and show how the same mechanism can induce the first
    best outcome in both cases.
  \end{itemize}
\end{frame}

\section{Jump state inconsistency}

\subsection{Model}

\begin{frame}
  \frametitle{The model}
  A model of a durable good producing monopolist who pollutes is used
  for exposition.
  \begin{itemize}
  \item<1-> Heterogeneous consumers each purchase one unit of a
    durable good that gives them a stream of benefits in perpetuity.
  \item<2-> Over time the price of the durable good declines as the
    keenest consumers purchase it.
  \item<3-> Consumers decide when to purchase the good and exit the
    market. He weigh the decreasing price against his valuation of the
    good in order to make his decision.
  \item<4-> Thus, consumer demand depends upon his expectations of
    future prices. Consequently, the monopolist's profits and the
    welfare function also depend upon expected future prices.
  \end{itemize}
\end{frame}


\subsection{Benchmarks}

\begin{frame}
  \frametitle{Benchmarking the regulator}
  Suppose that the regulator can directly choose prices in the
  market. The first-best price path is the one that a regulator who
  could perfectly precommit would choose in order to maximise welfare.

  In this case, the regulator would solve
  \begin{equation}\label{eq:1}
    \max_{\mathbf{p}^t} \sum^\infty_{t=1} \delta^{t-1}
    w(p^{t-1},p^t,p^{t+1}_e).
  \end{equation}
\end{frame}

\begin{frame}
  The solution to this problem will satisfy the following DEs
  \begin{align}
    w^t_2 + \delta w^{t+1}_1 & = 0, \quad t=1 \label{four}\\
    w^{t-1}_3 + \delta w^t_2 + \delta^2 w^{t+1}_1 & = 0, \quad t \geq
    2 \label{five}.
  \end{align}
  The source of the time-inconsistency in is the term $w^{t-1}_3$,
  which shows the effect of a change in the current period's price on
  the previous period's welfare.

  When optimisation occurs only the future periods' welfare is
  considered and the effect on the previous period is disregarded.
\end{frame}

\begin{frame}
  \frametitle{Second-best regulation}
  If the regulator is unable to precommit then he can still act in a
  time consistent fashion by anticipating his future deviations. He
  will solve the Bellman equation
  \begin{equation} \label{tc:bell} V(p^{t-1}) = \max_{p^t} \Bigl\{ w
    \bigl( p^{t-1},p^t,f_e(p^t) \bigr) + \delta V(p^t) \Bigr\} \qquad
    \forall\: t \geq 1
  \end{equation}
    for an equilibrium strategy, $f(p^t)$, that satisfies
  \begin{equation} \label{tc:mpe} w^t_2 + w^t_3 f^{t+1}_1 + \delta
    w^{t+1}_1 = 0.
\end{equation}
\end{frame}


\begin{frame}
    \frametitle{Inefficiency}
    \begin{itemize}
    \item<1-> The time consistent regulator realises that he will
      have an incentive to deviate in future, so he sets the current
      price such that the incentive is removed.
    \item<2-> Constraining the regulator to be time consistent forces
      him to deviate from the first-best price path, even when he has
      direct control over the good's pricing.
    \item<3-> If profits are a large part of total welfare then this
      is likely to result in too much pollution, as the price path
      will be inefficiently low.
    \end{itemize}
\end{frame}

\subsection{The solution}

\begin{frame}
  \frametitle{Delegation and precommitment}
  \begin{itemize}
  \item<1-> The value of precommitment in games has long been
    recognised. The difficulty is in finding an effective
    precommitment mechanism.
  \item<2-> Work on strategic delegation has focussed on using a
    manager as a commitment tool. By contracting with a manager the
    form's owner can remove themselves from the current period's
    output decision.
  \item<3-> The key aspect of the delegation is that an agent is paid
    to make the current decision for you. The sum you pay him now thus
    credibly commits the owner to a particular decision in the next
    period.
  \item<4-> Here we can achieve the same thing by using the regulated
    polluter as a commitment tool for the regulator. The 'payment'
    here is the tax that is levied on the polluter.
  \end{itemize}
\end{frame}

\begin{frame}
  \frametitle{Taxation}
  \begin{itemize}
  \item<1-> The regulator now levies a tax $\tau^t$ on the
    monopolist's output/emissions, so the monopolist's profit becomes
    dependent upon the tax rate: $\pi (p^{t-1}, p^t, p^{t+1}, \tau^t)$.
  \item<2-> The regulator receives a revenue valued at $\alpha\tau^t
    \psi(x^t)$ from the revenues. The tax rate can be altered each
    period, but there is a cost to doing so of
    $\theta\big(\tau^t-\tau^{t-1}\big)^2$. Thus, the welfare function
    is dependent upon current and past taxes:
    $w^t\big(p^{t-1},p^t,p^{t+1},\tau^{t-1},\tau^t\big)$.
  \item<3-> There is still a jump state in the welfare function, but
    the regulator now chooses $\tau^t$ when maximising welfare, and
    leaves the choice of price to the monopolist.
  \end{itemize}
\end{frame}

\begin{frame}
  \frametitle{The game}
  \begin{itemize}
  \item<1-> The decision of the regulator is modelled as an
    intra-personal game between the regulator's various temporal selves.
  \item<2-> We use the MPE solution concept for two reasons: it
    removes the possibility of trigger strategies, and it gives us
    time consistent policies.
  \end{itemize}
\end{frame}

\begin{frame}
    \begin{itemize}
\item<1-> Let $\tau^t = f(p^{t-1}, \tau^{t-1})$ be the regulator's
    period-$t$ taxation strategy and $p^t = g(p^{t-1}, \tau^{t-1})$ be
    the monopolist's period-$t$ pricing strategy.
  \item<2-> In equilibrium, the regulator solves the Bellman equation
    \scriptsize{\begin{multline} 
      W(p^{t-1},\tau^{t-1}) = \max_{\tau^t} \Bigg\{ w \bigg( p^{t-1},
      g\big(p^{t-1},\tau^{t-1}\big), g \Big(
      g\big(p^{t-1},\tau^{t-1}\big) , \tau^t \Big) \tau^{t-1}, \tau^t
      \bigg) \\ + \delta W \Big( g\big(p^{t-1},\tau^{t-1}\big) ,
      \tau^t \Big) \Bigg\}
\end{multline}}
\end{itemize}
\end{frame}

\begin{frame}
  \frametitle{Equilibrium strategies}
  \begin{itemize}
  \item<1-> The key to this formulation is that the welfare value
    function no longer depends upon any future decision of the
    regulator, since the future decision has been delegated out.
  \item<2-> The regulator thus chooses his tax rate such that the
    price path will replicate the first-best outcome and pollution
    levels will be optimal.
  \item<3-> If there is a positive cost to policy adjustment, or the
    regulator and monopolist value tax revenue differently, then the
    price path will be distorted away from the first-best.
  \end{itemize}
\end{frame}

\begin{frame}
  \frametitle{Key assumptions}
  This result depends on some important assumptions
  \begin{itemize}
  \item<1-> Rational expectations;
  \item<2-> Simultaneous decision making.
  \end{itemize}
\end{frame}


\section{Hyperbolic discounting}

\subsection{Framework}

\begin{frame}
  \frametitle{A quasi-hyperbolic formulation}
  \begin{itemize}
  \item<1-> Most authors who work with hyperbolic discounting use a
    'quasi-hyperbolic' discount rate where the current period's
    discount rate is $\beta\delta$ and the discount rate between
    future periods is $\delta$.
  \item<2-> Agents are thus time inconsistent between the current and
    future decisions, but are time consistent across expected future decisions.
  \end{itemize}
\end{frame}

\subsection{Implementing delegation}

\begin{frame}
  \frametitle{Delegation}
  \begin{itemize}
  \item<1-> A model of a polluting monopolist in the context of a
    regulator with quasi-hyperbolic preferences was constructed to
    explore whether a similar delegation mechanism would work in this setting.
  \item<2-> Since the regulator is only time inconsistent between the
    current and next periods, delegation easily solves this problem
    and renders the price path time consistent.
  \item<3-> This model also depends on simultaneous decision making,
    but the quasi-hyperbolic formulation is also essential. A truly
    hyperbolic discount rate would not be amenable to such a solution.
  \end{itemize}
\end{frame}


\section{Conclusions}

\begin{frame}
  \frametitle{Take-home lessons}
  \begin{itemize}
  \item<1-> Regulators (might) need to think about the time
    consistency of his decisions if he want to save the world from
    global warming.
  \item<2-> Overcoming time inconsistency doesn't need to involve
    external precommitment mechanisms. First best outcomes can be
    achieved within the regulator/polluter relationship if the
    institutional framework is correctly structured.
  \item<3-> The key to the framework we propose is controlling the
    information that parties have when he make his decisions. The
    monopolist can't know the current tax rate when he make his
    decision.
  \end{itemize}
\end{frame}

\end{document}
